% Algemene settings
\documentclass{beamer}
\usepackage[dutch]{babel}
\usepackage{graphicx}
\graphicspath{{img/}}


% Thema en stijl
\usetheme{Madrid}
\usecolortheme{whale}
\beamertemplatenavigationsymbolsempty
\setbeamertemplate{itemize items}[circle] 


% ToC aan begin elke subsectie (indien te vaak verander
% naar section)
\AtBeginSection[]
{
  \begin{frame}<beamer>{Inhoud}
  \small
    \tableofcontents[currentsection]
  \end{frame}
}


% Title page
\title[Capaciteitsproblemen]{\Huge{Capaciteitsproblemen bij online ticketverkoop}}
\author[Groep 2]{Sonia Amiri \and Rami Berro \and Florian Braùn \and Tom Cottem \and Thijs Creemers \and John Cai \and Lode Dockx}
\date[IW]{\small{Informaticawerktuigen - 29 november 2023}}
\logo{\includegraphics[height = 1cm]{kul-logo.png}}


% Start slides
% TEKST VAN ELK DEEL MAG 1m20 DUREN
\begin{document}

\begin{frame}
  \titlepage
\end{frame}


\section[Intro]{Introductie}
% TEKST (draft)
% ==============
% Begroeting.
% Zeggen waarover de presentatie gaat.
% Vermelden aanleiding:
    % In 2022 ontstond ophef vanwege de chaotische online ticketverkoop voor 
    % Taylor Swift-concerten door Ticketmaster. Het Amerikaanse bedrijf werd
    % geconfronteerd met capaciteitsproblemen doordat veel fans tegelijkertijd
    % zijn website wilden raadplegen.
% Uitleggen dat vaker voorkomt bij online diensten, zowel door normale pieken als door zaken zoals DoS-aanvallen.
% Idee dat dit belangrijke gevolgen heeft: kosten, ontevredenheid
% We zullen daarom dit, dit en dat bespreken.
\begin{frame}{Introductie}
    \begin{itemize}
        \item Chaos bij ticketverkoop Swift in 2022
        \item Capaciteitsproblemen bij Ticketmaster
        \item Geen alleenstaand geval
        \item Mogelijke oorzaken: normaal gebruik site/app, gebruikspieken, cyberaanval ...
        \item Mogelijke gevolgen: inkomstenverlies, imagoschado, ontevredenheid, veiligheidsrisico's ...
    \end{itemize}
    \begin{center}
        \includegraphics[height = 2.5cm]{ticketmaster-logo.png}
        \includegraphics[height = 2.5cm]{taylor-swift.png}
        \includegraphics[height = 2.5cm]{server-rack.jpg}
    \end{center}
\end{frame}


\begin{frame}{Inhoud}
  \tableofcontents
\end{frame}


\section[Actualiteitsproblemen]{Actualiteitsproblemen ticketsector}
% TEKST (draft)
% ==============
% Ticketmaster werd recentelijk bekritiseerd door de uitverkoop van Taylor Swift's Eras Tour in november 2022. 
% Deze tour werd voorspeld als een van de best verkochte in de geschiedenis, als de comeback na Swift's eerdere succesvolle
% Reputation-wereldtournee. Ticketmaster meldde meer dan 14 miljoen bezoekers op hun site tijdens de voorverkoop, maar beweerde slechts
% 1,5 miljoen toegangscodes te hebben uitgegeven. Fans waren verontwaardigd over lange wachttijden, een haperend systeem, tekort aan
% tickets en abrupte prijsstijgingen. Politiek gezien riep vertegenwoordiger Alexandria Ocasio-Cortez op tot verzet tegen Ticketmaster
% en benoemde het een monopolie door de fusie met LiveNation, waarbij de enorme prijsstijgingen en wachtrijen als voornaamste redenen
% werden aangehaald. Met meer dan 70% controle op de markt, kan Ticketmaster zijn macht misbruiken zonder concurrentie, zoals duidelijk
% werd tijdens de uitverkoop van de Eras Tour.
\begin{frame}{Actualiteitsproblemen ticketsector}
    \begin{itemize}
        \item Taylor Swift - Eras Tour
        \item Verwachtte best verkochte tour
        \item Fans woedend door lange wachttijden, hoge prijzen ...
        \item Politieke backlash
        \item Kritiek op marktpositie Ticketmaster
    \end{itemize}
\end{frame}

\section[Capaciteitsproblemen]{Capaciteitsproblemen bij overbelasting online diensten}
\begin{frame}{Capaciteitsproblemen bij overbelasting online diensten}
    \begin{itemize}
        \item Overbevraging
        \item Capaciteitstekorten
        \item Trage systemen
        \item Slecht functionerende systemen
        \item Niet functionerende systemen
        \item Pieken
        \item Normale pieken
        \item Onwenselijke pieken
        \item Malafide pieken
    \end{itemize}
\end{frame}

\section[Bedrijf]{Belangen van ticketsysteembedrijven}
\begin{frame}{Belangen van ticketsysteembedrijven}
    \begin{itemize}
        \item Marktsituatie en concurrentie
        \item Winstmaximalisatie
        \item Imago en merkreputatie
    \end{itemize}
\end{frame}

    \subsection{}
    \begin{frame}{Marktsituatie en concurrentie}
        \begin{columns}
            \begin{column}{.4\textwidth}
                \begin{itemize}
                    \item Primaire ticketverkoper
                    \item Secundaire markt
                \end{itemize}
            \end{column}
            \begin{column}{.6\textwidth}
                \begin{figure}
                    \includegraphics[width=50px,height=50px,keepaspectratio]{ticketmaster-logo.png}
                \end{figure}
                \centering
                \includegraphics[width=45px,height=45px,keepaspectratio]{stubhub-logo.png}
                \includegraphics[width=45px,height=45px,keepaspectratio]{eventbrite-logo.png}
            \end{column}
        \end{columns}
    \end{frame}
    
    \subsection{}
    \begin{frame}{Winstmaximalisatie?}
        \begin{itemize}
            \item Andere sites = negatief?
            \item Live Nation
        \end{itemize}
    \end{frame}

    \subsection{}
    \begin{frame}{Imago en merkreputatie}
        \begin{itemize}
            \item Maatregelen tegen misbruik secundaire markt
            \item Hoe het imago behouden?
        \end{itemize}
    \end{frame}
% Als we het hebben over de belangen van ticketsysteembedrijven, draait het voornamelijk om de marktsituatie, concurrentie, 
    % winstmaximalisatie en het behoud van imago en merkreputatie.

% Een dominante speler in de primaire ticketverkoop is Ticketmaster. Als primaire marktverkoper betekent dit dat 
    % zij tickets rechtstreeks verkopen die aan hun zijn toegewezen door evenementpartners. Aan de andere kant, in de 
    % secundaire markt, verkopen concurrenten zoals Eventbrite en StubHub reeds verkochte primaire tickets, meestal aan een hogere prijs.

% Maar hoe streven deze bedrijven dan naar winstmaximalisatie? Hoewel concurrerende sites op het eerste gezicht mogelijk als 
    % negatief worden beschouwd, is het tegendeel waar. Bijvoorbeeld, Live Nation, het moederbedrijf van Ticketmaster, heeft 
    % juist de secundaire markt niet tegengewerkt, maar eerder een bedrijf op die markt, zoals TicketsNow, overgenomen.
    
% Wanneer we spreken over imago en merkreputatie, kan een dergelijke actie leiden tot controverse, vooral omdat Ticketmaster 
    % zelf beweerde maatregelen te nemen tegen de doorverkoop van tickets door de secundaire markt. Het behoud van hun imago is
    % natuurlijk essentieel, en dit wordt bereikt door bijvoorbeeld afspraken na te komen met externe partijen zoals artiesten.

\section[Klant]{Belangen van klanten/gebruikers}
\begin{frame}{Belangen van klanten/gebruikers}
    \begin{itemize}
        \item Voordelen 
        \item Nadelen
    \end{itemize} 
\end{frame}
    \subsection{}
    \begin{frame}{Voordelen}
        \begin{itemize}
            \item Gebruiksvriendelijkheid
            \item Helpdesk
            \item Beschikbaarheid
        \end{itemize}
    \end{frame}
            
    \subsection{}
    \begin{frame}{Nadelen}
        \begin{itemize}
            \item Technische aanleg
            \item Crashes
            \item Prijs
        \end{itemize}
    \end{frame}
    % Ticketsystemen hebben ten opzichte van de gebruiker voordelen en nadelen. Dit komt doordat er veel factoren komen kijken bij het maken van ticketsystemen. 
    % deze voor- en nadelen kunnen de belangen van een klant toe- en tegenwerken.
    
    % De voordelen van een ticketsysteem zijn redelijk evident, maar zeker niet altijd heel merkbaar. Zo zijn ticketsystemen zeer gebruiksvriendelijk,
    % ze worden ontworpen zodaning dat gebruikers zo gemakkelijk en efficiënt mogelijk te werk kan gaan en hun tickets kunnen aankopen. 
    % Als dat dan toch niet vlot of goed verloopt hebben meeste ticketsystemen en helpdesk waar de klanten meestal 24/7 terecht kunne komen voor hulp.
    % ticketsystemen zijn ook altijd online beschikbaar dit is heel handig, klanten kunnen altijd hun tickets controleren en als ze er nog geen hebben kunnen
    % ze soms zelfs last minute tickets aankopen.
    
    % Ookal worden ticketsystemen makkelijk hebben gebruikers toch enige technische aanleg nodig, sommige gebruikers hebben misschien moeite met het gebruik van online systemen
    % Voor online systemen is ook een (stabiele) internetverbinding nodig, zonder kan het wel eens foutlopen en ookal heb je een (stabiele) internetverbinding kan het soms 
    % nog zijn dat de site crashed doordat er teveel mensen tegelijk op zijn, waardoor tickets aankopen dus niet gaat lukken 
    % ALs je dan wel een ticket kan aankopen kan het dan alsnog zijn dat de prijs flink kan oplopen, dit door commissies en fees die de ticketing maatschapij kan opleggen.
    % Veel ticketsystemen hebben ook een slechte of geen refund policy waardoor de klant/gebruiker geld verliest.
    
    

    
\section[Machtsmisbruik]{Machtsmisbruik door grote spelers}
\begin{frame}{Machtsmisbruik door grote spelers}
    \begin{itemize}
        \item Monopolie
        \begin{itemize}
            \item Tickets
            \item Informatie
            \item Verkoopkanalen
            \item Promotie
            \item Locaties
        \end{itemize}
        \item Prijsstelling en commissies
        \item Exclusieve Deals
        \item Beperking van Keuze voor Consumenten
        \item Data Verzameling en Privacy
  \end{itemize}
\end{frame}

% Doordat de grote spelers zoveel macht hebben is er en monopolie ontstaan, dit monopolie heeft een grote invloed op de markt en de consumenten.
% Zo hebben de grote spelers een monopolie op de tickets, ze zijn de enige die de tickets kunnen verkopen en de prijs kunnen bepalen.
% Ook hebben ze een monopolie op de informatie, ze zijn de enige die de informatie over de tickets hebben en kunnen bepalen wat er met de tickets gebeurt.
% De grote spelers hebben ook een monopolie op de verkoopkanalen, ze zijn de enige die de tickets kunnen verkopen en de prijs kunnen bepalen.
% Ook hebben ze een monopolie op de promotie, ze zijn de enige die de tickets kunnen verkopen en de prijs kunnen bepalen.
% De grote spelers hebben ook een monopolie op de locaties, ze zijn de enige die de tickets kunnen verkopen en de prijs kunnen bepalen.
% Door hun macht kunnen de grote spelers ook de prijs bepalen en commissies opleggen, dit is een groot probleem voor de consumenten.
% De grote spelers sluiten ook exclusieve deals af, dit is een groot probleem voor de consumenten.
% De grote spelers beperken ook de keuze voor de consumenten, dit is een groot probleem voor de consumenten.
% De grote spelers verzamelen ook data en schenden de privacy van de consumenten, dit is een groot probleem voor de consumenten.
    
\section[Maatregelen]{Maatregelen tegen machtsmisbruik}
\begin{frame}{Maatregelen tegen machtsmisbruik}
    \begin{itemize}
        \item Wetgeving
        \item Alternatieve verkoopkanalen
        \item Alternatieve promotiekanalen
        \item Alternatieve locaties
    \end{itemize}
\end{frame}
   
\section{Conclusie}
\begin{frame}{Conclusie}
%Vandaag werden de capaciteitsproblemen besproken=>  aantal oplossingen hiervoor gegeven zoals verticaal en horizontaal opschalen. De aanwezigheid van machtsmisbruik werd duidelijk gemaakt=> regulering, transparantie en het exploreren van alternatieve verkoopkanalen als potentiele oplossingen tegen machtsmisbruik en een eerlijkere omgeving voor alle partijen . dus niet enkel technische problemen maar ook=> Marktdominantie, consumentenbelangen en noodzaak van regulering. oplossing?=> concurrentie, consumentenbescherming en transparantie in de ticketverkoopindustrie.
    \begin{itemize}
        \item Machtsmisbruik 
        \item Maatregelen
        \item Alternatieven 
    \end{itemize}
\end{frame}

\end{document}
