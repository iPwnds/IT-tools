\documentclass{beamer}
\usepackage[dutch]{babel}
\usepackage{graphicx}

\usetheme{Singapore}
\graphicspath{{img/}}

\title{\Huge{Capaciteitsproblemen bij online ticketverkoop}}
\author{{Sonia Amiri \and Rami Berro \and Florian Braùn \and Tom Cottem \and Thijs Creemers \and John Cai \and Lode Dockx}}
\date{\small{Informaticawerktuigen - 29 november 2023}}

% Delete this, if you do not want the table of contents to pop up at
% the beginning of each subsection:
\AtBeginSubsection[]
{
  \begin{frame}<beamer>{Inhoud}
  \small
    \tableofcontents[currentsection]
  \end{frame}
}

\begin{document}

\begin{frame}
  \titlepage
\end{frame}

\begin{frame}{Inhoud}
  \tableofcontents
\end{frame}

\section[Intro]{Introductie}
\begin{frame}{Introductie}
%In 2022 ontstond ophef vanwege de chaotische online ticketverkoop voor 
%Taylor Swift-concerten door Ticketmaster. Het Amerikaanse bedrijf werd
%geconfronteerd met capaciteitsproblemen doordat veel fans tegelijkertijd
%zijn website wilden raadplegen. Die storingen hebben ook gevolgen, namelijk ontevredenheid en die van verschillende aard zijn zolas inkomstenverlies bij onbereikbaarheid e-commerceplatformen eiligheidskosten bij ontoegankelijkheid informatie tijdens rampen enz. Daarom gaan we vandaag drie belangrijke vragen beantwoorden. "Wat zijn capaciteitsproblemen en hun mogelijke oplossingen?",
%"Hoe worden belangen van bedrijven en klanten geïmpacteerd door de marktsituatie en capaciteitsproblemen?" en "In welke mate is er economisch machtsmisbruik en welke maatregelen kunnen dat verhinderen?"
    \begin{itemize}
        \item Taylor Swifts concert
        \item Capaciteitsproblemen
        \item Gevolgen
        \item Oplossingen
        \item Belangen van bedrijven en klanten
        \item Economische machtsmisbruik
    \end{itemize}
\end{frame}

\section[Casus Taylor]{Casus: Taylor Swift} %??
\begin{frame}{Casus: Taylor Swift}
    \begin{itemize}
        \item Taylor swifts Eras Tour
        \item Verwachte best verkochte tour
        \item Fans woedend: lange wachttijden, hoge prijzen
        \item Politieke backlash
        \item Monopolische macht van Ticketmaster
    \end{itemize}
% Ticketmaster werd recentelijk bekritiseerd door de uitverkoop van Taylor Swift's Eras Tour in november 2022. 
%Deze tour werd voorspeld als een van de best verkochte in de geschiedenis, als de comeback na Swift's eerdere succesvolle
% Reputation-wereldtournee. Ticketmaster meldde meer dan 14 miljoen bezoekers op hun site tijdens de voorverkoop, maar beweerde slechts
% 1,5 miljoen toegangscodes te hebben uitgegeven. Fans waren verontwaardigd over lange wachttijden, een haperend systeem, tekort aan
% tickets en abrupte prijsstijgingen. Politiek gezien riep vertegenwoordiger Alexandria Ocasio-Cortez op tot verzet tegen Ticketmaster
% en benoemde het een monopolie door de fusie met LiveNation, waarbij de enorme prijsstijgingen en wachtrijen als voornaamste redenen
% werden aangehaald. Met meer dan 70% controle op de markt, kan Ticketmaster zijn macht misbruiken zonder concurrentie, zoals duidelijk
% werd tijdens de uitverkoop van de Eras Tour.
\end{frame}

\section[Capaciteit]{Capaciteitsproblemen bij overbevraging}
\begin{frame}{Capaciteitsproblemen bij overbevraging}
    \begin{itemize}
        \item Overbevraging
        \item Capaciteitstekorten
        \item Trage systemen
        \item Slecht functionerende systemen
        \item Niet functionerende systemen
        \item Pieken
        \item Normale pieken
        \item Onwenselijke pieken
        \item Malafide pieken
    \end{itemize}
\end{frame}

\section[Bedrijf]{Belangen van ticketsysteembedrijven}
\begin{frame}{Belangen van ticketsysteembedrijven}
    \begin{itemize}
        \item Marktsituatie en concurrentie
        \item Winstmaximalisatie
        \item Imago en merkreputatie
    \end{itemize}
\end{frame}

    \subsection{}
    \begin{frame}{Marktsituatie en concurrentie}
        \begin{itemize}
            \item Primaire ticketverkoper
            \item Secundaire markt
        \end{itemize}
        \begin{figure}
            \includegraphics[width=30px,height=30px,keepaspectratio]{ticketmaster-logo}            
        \end{figure}
    \end{frame}
    
    \subsection{}
    \begin{frame}{Winstmaximalisatie?}
        \begin{itemize}
            \item Andere sites = negatief?
            \item Live Nation
        \end{itemize}
    \end{frame}

    \subsection{}
    \begin{frame}{Imago en merkreputatie}
        \begin{itemize}
            \item Maatregelen tegen misbruik secundaire markt
            \item Hoe het imago behouden?
        \end{itemize}
    \end{frame}
% Als we het hebben over de belangen van ticketsysteembedrijven, draait het voornamelijk om de marktsituatie, concurrentie, 
    % winstmaximalisatie en het behoud van imago en merkreputatie.

% Een dominante speler in de primaire ticketverkoop is Ticketmaster. Als primaire marktverkoper betekent dit dat 
    % zij tickets rechtstreeks verkopen die aan hun zijn toegewezen door evenementpartners. Aan de andere kant, in de 
    % secundaire markt, verkopen concurrenten zoals Eventbrite en StubHub reeds verkochte primaire tickets.

% Maar hoe streven deze bedrijven dan naar winstmaximalisatie? Hoewel concurrerende sites op het eerste gezicht mogelijk als 
    % negatief worden beschouwd, is het tegendeel waar. Bijvoorbeeld, Live Nation, het moederbedrijf van Ticketmaster, heeft 
    % juist de secundaire markt niet tegengewerkt, maar eerder een bedrijf op die markt, zoals TicketsNow, overgenomen.
    
% Wanneer we spreken over imago en merkreputatie, kan een dergelijke actie leiden tot controverse, vooral omdat Ticketmaster 
    % zelf beweerde maatregelen te nemen tegen de doorverkoop van tickets door de secundaire markt. Het behoud van hun imago is
    % natuurlijk essentieel, en dit wordt bereikt door bijvoorbeeld afspraken na te komen met externe partijen zoals artiesten.

\section[Klant]{Belangen van klanten/gebruikers}
\begin{frame}{Belangen van klanten/gebruikers}
    \begin{itemize}
        \item Voordelen 
        \item Nadelen
    \end{itemize} 
\end{frame}
    \subsection{}
    \begin{frame}{Voordelen}
        \begin{itemize}
            \item Gebruiksvriendelijkheid
            \item Helpdesk
            \item Beschikbaarheid
        \end{itemize}
    \end{frame}
            
    \subsection{}
    \begin{frame}{Nadelen}
        \begin{itemize}
            \item Technische aanleg
            \item Crashes
            \item Prijs
        \end{itemize}
    \end{frame}
    % Ticketsystemen hebben ten opzichte van de gebruiker voordelen en nadelen. Dit komt doordat er veel factoren komen kijken bij het maken van ticketsystemen. 
    % deze voor- en nadelen kunnen de belangen van een klant toe- en tegenwerken.
    
    % De voordelen van een ticketsysteem zijn redelijk evident, maar zeker niet altijd heel merkbaar. Zo zijn ticketsystemen zeer gebruiksvriendelijk,
    % ze worden ontworpen zodaning dat gebruikers zo gemakkelijk en efficiënt mogelijk te werk kan gaan en hun tickets kunnen aankopen. 
    % Als dat dan toch niet vlot of goed verloopt hebben meeste ticketsystemen en helpdesk waar de klanten meestal 24/7 terecht kunne komen voor hulp.
    % ticketsystemen zijn ook altijd online beschikbaar dit is heel handig, klanten kunnen altijd hun tickets controleren en als ze er nog geen hebben kunnen
    % ze soms zelfs last minute tickets aankopen.
    
    % Ookal worden ticketsystemen makkelijk hebben gebruikers toch enige technische aanleg nodig, sommige gebruikers hebben misschien moeite met het gebruik van online systemen
    % Voor online systemen is ook een (stabiele) internetverbinding nodig, zonder kan het wel eens foutlopen en ookal heb je een (stabiele) internetverbinding kan het soms 
    % nog zijn dat de site crashed doordat er teveel mensen tegelijk op zijn, waardoor tickets aankopen dus niet gaat lukken 
    % ALs je dan wel een ticket kan aankopen kan het dan alsnog zijn dat de prijs flink kan oplopen, dit door commissies en fees die de ticketing maatschapij kan opleggen.
    % Veel ticketsystemen hebben ook een slechte of geen refund policy waardoor de klant/gebruiker geld verliest.
    
    

    
\section[Machtsmisbruik]{Machtsmisbruik door grote spelers}
\begin{frame}{Machtsmisbruik door grote spelers}
    \begin{itemize}
        \item Monopolie op tickets
        \item Monopolie op informatie
        \item Monopolie op verkoopkanalen
        \item Monopolie op promotie
        \item Monopolie op locaties
    \end{itemize}
\end{frame}
    
\section[Maatregelen]{Maatregelen tegen machtsmisbruik}
\begin{frame}{Maatregelen tegen machtsmisbruik}
    \begin{itemize}
        \item Wetgeving
        \item Alternatieve verkoopkanalen
        \item Alternatieve promotiekanalen
        \item Alternatieve locaties
    \end{itemize}
\end{frame}
    
\section{Conclusie}
\begin{frame}{Conclusie}
    \begin{itemize}
        \item Machtsmisbruik door grote spelers
        \item Maatregelen tegen machtsmisbruik
        \item Alternatieven voor ticketsysteem
    \end{itemize}
\end{frame}
    
\section{Ref}
\begin{frame}{Referenties}
%     \begin{itemize}
%         \item https://www.similarweb.com/top-websites/e-commerce-and-shopping/tickets/
%         \item https://en.wikipedia.org/wiki/Ticketmaster
%         \item https://www.investopedia.com/is-ticketmaster-a-monopoly-6834539#:~:text=What\%20is\%20Ticketmaster\%27s\%20market\%20share,by\%20far\%20the\%20industry\%20leader.
%     \end{itemize}
\end{frame}

\section[Appendix]{Appendix over GAI}
\begin{frame}{Appendix over GAI}
    \begin{itemize}
        \item GAI
    \end{itemize}
\end{frame}

\end{document}
