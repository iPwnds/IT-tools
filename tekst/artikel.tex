% Variabelen gebruikt in template.
\newcommand{\titletext}{Capaciteitsproblemen bij online ticketverkoop}
\newcommand{\subtitletext}{Oorzaken, gevolgen en oplossingen}
\newcommand{\logopath}{img/kul-logo.png}
\newcommand{\groupmembera}{Sonia Amiri}
\newcommand{\groupmemberb}{Rami Berro}
\newcommand{\groupmemberc}{Florian Braùn}
\newcommand{\groupmemberd}{John Cai}
\newcommand{\groupmembere}{Tom Cottem}
\newcommand{\groupmemberf}{Thijs Creemers}
\newcommand{\groupmemberg}{Lode Dockx}
\newcommand{\groupnumber}{2}
\newcommand{\academicyear}{2023-2024}

% Artikel gebaseerd op LaTeX-template KUL (gevonden op Overleaf).
\documentclass[a4paper, 12pt]{article}

% Packages gebruikt door LaTeX-template KUL.
\usepackage[dutch]{babel}
\usepackage{graphicx}
\usepackage[colorlinks, linkcolor=black, citecolor=black, urlcolor=black]{hyperref}
\usepackage{geometry}
\geometry{tmargin=3cm, bmargin=2.2cm, lmargin=2.2cm, rmargin=2cm}
% Change marginparwidth to prevent issues with todonotes.
\setlength{\marginparwidth}{2cm}
\usepackage{todonotes}
\usepackage{ifthen}

% Zelf toegevoegde packages.
\usepackage{verbatim} % Voor gebruik van multi-line comments.

\begin{document}

% Voor geanonimiseerd rapport te maken (onderdeel template, irrelevant).
\newboolean{anonymize}
% Onderstaande niet uncommenten, anders geanonimiseerd rapport.
% \setboolean{anonymize}{true}

% Template titlepage.
\begin{titlepage}
    \newpage
    \thispagestyle{empty}
    \frenchspacing
    \hspace{-0.2cm}
    \includegraphics[height=3.4cm]{\logopath}
    \hspace{0.2cm}
    \rule{0.5pt}{3.4cm}
    \hspace{0.2cm}
    \begin{minipage}[b]{8cm}
        \Large{Katholieke\newline Universiteit\newline Leuven}\smallskip\newline
        \large{}\smallskip\newline
        \textbf{Departement\newline Computerwetenschappen}\smallskip
    \end{minipage}
    \hspace{\stretch{1}}
    \vspace*{3.2cm}\vfill
    \begin{center}
        \begin{minipage}[t]{\textwidth}
            \begin{center}
                \LARGE{\rm{\textbf{\uppercase{\titletext}}}}\\
                \Large{\rm{\subtitletext}}
            \end{center}
        \end{minipage}
    \end{center}
    \vfill
    \hfill\makebox[8.5cm][l]{
        \vbox to 7cm{\vfill\noindent
            \ifthenelse{\boolean{anonymize}}{
                {\rm \textbf{Anonymized}}\\
                {\rm \academicyear}
            }{
                {\rm \textbf{Groep \groupnumber}}\\
                {\rm {\groupmembera}}\\
                {\rm {\groupmemberb}}\\
                {\rm {\groupmemberc}}\\
                {\rm {\groupmemberd}}\\
                {\rm {\groupmembere}}\\
                {\rm {\groupmemberf}}\\
                {\rm {\groupmemberg}}\\[2mm]
                {\rm \academicyear}

            }
        }
    }
\end{titlepage}

% Inhoudstafel.
\tableofcontents
\newpage

% Invoeging secties uit aparte bestanden.
% Hierin vertel je waarover het artikel gaat en tracht je de interesse van de
% lezer te wekken. Het einde van de inleiding
% geeft een kort overzicht van hoe het artikel gestructureerd is. 

\newpage
% De verschillende secties moeten mooi overgaan in elkaar.
% Hiervoor gebruik je bindteksten. Deze tekstjes zijn typisch de eerste
% paragrafen van de verschillende (sub)secties en leggen uit wat de
% bedoeling van de sectie is, en eventueel hoe dat relateert aan
% de vorige sectie of hoe het past in het grotere geheel.

\section{Ticketsystemen}

\subsection{Wat is een ticketsysteem?}
 Een ticketsysteem is een geavanceerde softwaretoepassing of mechanisme dat een essentiële rol speelt bij het stroomlijnen van het beheer en de tracking van verschillende soorten verzoeken, incidenten of problemen. Deze verzoeken, vaak aangeduid als "tickets", kunnen een breed scala van categorieën omvatten, zoals klantvragen, technische ondersteuningsverzoeken, onderhoudstaken, serviceaanvragen en meer. In de kern fungeert een ticketsysteem als een gecentraliseerde hub waar individuen of gebruikers hun verzoeken indienen, en organisaties deze efficiënt kunnen afhandelen, prioriteren en oplossen.

Ticketsystemen zijn ontworpen om orde en structuur te brengen in het vaak complexe proces van het beheer van inkomende verzoeken. Ze bieden een gestructureerd kader voor het maken, toewijzen, volgen en oplossen van tickets, waardoor organisaties tijdig en efficiënte diensten of oplossingen kunnen bieden aan degenen die ze nodig hebben.

\subsection{Hoe werkt een ticketsysteem?}
 Een ticketsysteem werkt volgens een reeks goed gedefinieerde principes en processen, met de volgende belangrijke componenten en functionaliteit:
 
 \begin{description}
  \item[Ticketcreatie:] Wanneer een gebruiker of klant een probleem tegenkomt of een verzoek heeft, gebruiken ze meestal het ticketsysteem om een nieuw ticket te maken. Dit houdt in dat ze details verstrekken over het probleem of verzoek, zoals een beschrijving, relevante documenten en andere relevante informatie.

  \item[Toewijzing van Tickets:] De volgende stap van het systeem omvat het toewijzen van het ticket aan het juiste personeel of team dat verantwoordelijk is voor de afhandeling van het verzoek. Toewijzingen kunnen gebaseerd zijn op factoren zoals de aard van het ticket, de prioriteit ervan, of de beschikbaarheid en expertise van de toegewezen teamleden.

  \item[Tracking van Tickets:] Zodra toegewezen, volgt het systeem voortdurend de status en voortgang van elk ticket. Dit stelt zowel gebruikers als ondersteuningsteams in staat om de reis van het ticket van indiening tot oplossing te volgen. Transparantie en verantwoordelijkheid zijn essentiële aspecten van dit volgproces.

  \item[Communicatie en Samenwerking:] Ticketsystemen bevatten vaak ingebouwde communicatie- en samenwerkingstools. Deze functies stellen teamleden in staat om te discussiëren en informatie te delen met betrekking tot het ticket, zodat iedereen die betrokken is, goed geïnformeerd is en effectief kan samenwerken.

  \item[Prioriteit en Escalatie:] Tickets worden meestal gecategoriseerd en geprioriteerd op basis van hun urgentie en impact. Hoog-prioriteits-tickets kunnen sneller worden afgehandeld, terwijl complexe problemen mogelijk moeten worden geëscaleerd naar hogere niveaus van expertise of management voor oplossing.

  \item[Oplossing en Afsluiting:] Het uiteindelijke doel van een ticketsysteem is het vergemakkelijken van de oplossing van problemen of de uitvoering van verzoeken. Zodra een ticket is afgehandeld en het probleem is opgelost of de dienst is verleend, wordt het ticket als afgesloten gemarkeerd. Dit geeft de succesvolle voltooiing van het verzoek aan.
 \end{description}


\subsection{Waar kan een ticketsysteem gebruikt worden?}
 Ticketsystemen zijn opmerkelijk veelzijdig en vinden toepassingen in verschillende contexten en industrieën, waaronder:
 \begin{description}
   \item[Klantondersteuning:] In de zakelijke wereld vormen ticketsystemen een essentieel onderdeel van klantondersteuning. Ze dienen als het belangrijkste kanaal voor klanten om problemen te melden, hulp te vragen of vragen te stellen. Bedrijven gebruiken ticketsystemen om klantbehoeften te beheren, prioriteren en efficiënt aan te pakken, wat de klanttevredenheid verbetert.
   \item[IT Helpdesk:] Binnen organisaties vertrouwen IT-afdelingen vaak op ticketsystemen voor het beheer van IT-gerelateerde problemen en serviceverzoeken. Dit omvat het oplossen van technische problemen, het vervullen van verzoeken voor software of hardware, en het onderhouden van de digitale infrastructuur van de organisatie.
   \item[Onderhoud en Reparaties:] In industriële omgevingen worden ticketsystemen gebruikt om onderhouds- of reparatieverzoeken bij te houden en te beheren. Hierdoor wordt gegarandeerd dat apparatuur, machines of faciliteiten op de juiste manier worden onderhouden, wat de uitvaltijd minimaliseert en de operationele efficiëntie waarborgt.
   \item[Projectmanagement:] Sommige projectmanagement- en taakvolgsystemen hanteren ticketgebaseerde benaderingen om projecten en opdrachten te beheren. Deze methodologie maakt georganiseerd volgen van taken, hun status en voltooiing mogelijk.
   \item[Evenementenbeheer:] Evenementplanners en -organisatoren gebruiken ticketsystemen om logistiek te beheren en problemen met betrekking tot evenementen op te lossen. Dit kan toegangsbewijzen betreffen, het volgen van verzoeken met betrekking tot het evenement en het coördineren van de inspanningen van diverse teams.
  \end{description}
- Onderwijs: In onderwijsinstellingen worden ticketsystemen gebruikt voor een breed scala van administratieve en academische doeleinden. Ze kunnen worden ingezet voor het afhandelen van studentenvragen, het beheren van facilitair onderhoud en het volgen van verzoeken met betrekking tot academische ondersteuning.

\newpage
% De verschillende secties moeten mooi overgaan in elkaar.
% Hiervoor gebruik je bindteksten. Deze tekstjes zijn typisch de eerste
% paragrafen van de verschillende (sub)secties en leggen uit wat de
% bedoeling van de sectie is, en eventueel hoe dat relateert aan
% de vorige sectie of hoe het past in het grotere geheel.

\section{Casus: Taylor Swift}
\newpage
% De verschillende secties moeten mooi overgaan in elkaar.
% Hiervoor gebruik je bindteksten. Deze tekstjes zijn typisch de eerste
% paragrafen van de verschillende (sub)secties en leggen uit wat de
% bedoeling van de sectie is, en eventueel hoe dat relateert aan
% de vorige sectie of hoe het past in het grotere geheel.

\section{Belangen van ticketsysteembedrijven}



\subsection{Winstmaximalisatie}
Winstmaximalisatie is een cruciaal doel voor bedrijven in de ticketverkoopindustrie, en Ticketmaster is geen uitzondering. Om hun winst te maximaliseren, hanteert Ticketmaster vaak vijf belangrijke factoren bij het bepalen van de prijzen voor tickets. Hieronder worden deze factoren uitgelegd:

\vspace{5 mm}

Face Value Prijs: De face value prijs wordt beschouwd als de prijs die wordt toegepast door zowel de verkoper als de doorverkoper. Dit is een fluctuerende prijs die wordt beïnvloed door zowel vraag als aanbod. Ticketmaster maakt gebruik van dynamische prijsalgoritmen en processen om deze prijs te beïnvloeden op basis van deze factoren. Dit stelt Ticketmaster in staat om de prijs aan te passen aan de veranderende marktomstandigheden.

Service- en Orderverwerkingskosten: Ticketmaster voegt een servicekosten toe op basis van hun overeenkomst/contract met de artiest of de doorverkoper. Dit is een vaste prijs die niet verandert op basis van de nominale waarde van het werkelijke ticket zelf. Ze voegen ook een orderverwerkingskosten toe, die is ontworpen om de kosten van verzending en bescherming te dekken. De verwerkingskosten worden niet in rekening gebracht bij in-person aankopen. Eventuele resterende gelden uit de kosten worden verdeeld tussen de artiest en de verkoper.

Leveringskosten: Ticketmaster past deze prijs aan op basis van de gekozen leveringsmethode. Soms geldt deze vergoeding niet voor bepaalde aankopen. Deze vergoeding wordt extra inkomsten voor Ticketmaster.

Facilitatiekosten: De klant die zijn tickets op Ticketmaster verkoopt, beslist of deze vergoeding wordt toegepast. Deze kosten zijn bedoeld om winst terug te geven aan de daadwerkelijke locatie waar het evenement wordt gehouden. Ticketmaster ontvangt geen winst uit deze vergoeding.

Belastingen: Belastingen vormen over het algemeen geen belangrijke inkomstenbron in vergelijking met het totale inkomstenoverzicht van Ticketmaster. Deze worden bepaald door federale, staats- en lokale wetten en zijn niet onderhandelbaar bij aankoop.

\vspace{10 mm}

Het is duidelijk dat Ticketmaster een uitgebreid prijsmodel hanteert om winstmaximalisatie te bereiken. Door rekening te houden met deze factoren en hun dynamische prijsstrategie kunnen ze de prijzen aanpassen aan veranderende marktomstandigheden, waardoor ze zowel klanten als serviceproviders kunnen bedienen en tegelijkertijd streven naar winstoptimalisatie.

\subsection{Klanttevredenheid}

Het e-ticketconcept is aantrekkelijk voor zowel klanten als dienstverleners. Vanuit het perspectief van de klant biedt het de volgende voordelen:

A. Voordelen:
Snellere en handigere verificatie van een ticket;
De mogelijkheid om te profiteren van een flexibel tarievenschema, met mogelijke individuele kortingen en speciale aanbiedingen;
Mogelijkheid tot intrekking van verloren tickets en hun vervanging;
Geen noodzaak om contant geld bij zich te hebben, bijvoorbeeld voor een plaatselijke kaartjesautomaat. Dit is vooral handig voor klanten die de vervoersdienst slechts af en toe gebruiken of zich in een andere stad bevinden.

B. Nadelen:
Ondanks deze voordelen brengt het concept van e-ticketing ook enkele zorgen met zich mee, met name op het gebied van privacy. Deze zorgen omvatten:

Alomtegenwoordige klantidentificatie;
De mogelijkheid van klantprofilering, zoals het creëren van bewegingspatronen;
Potentiële schending van de privacy door verhoogd toezicht, wat vaak wordt aangeduid als "Het Grote Broer"-probleem.

Kortom, terwijl e-ticketing aanzienlijke voordelen biedt aan klanten, zijn er ook legitieme privacykwesties die moeten worden overwogen en aangepakt in de implementatie van dit systeem. Het handhaven van een balans tussen klanttevredenheid en privacybescherming is van cruciaal belang voor het succes van e-ticketingdiensten.

\subsection{Imago en merkreputatie}

In het streven naar winstmaximalisatie speelt het imago en de merkreputatie een essentiële rol voor bedrijven in de evenementenbranche, zoals Ticketmaster. Een positief imago en een sterke merkreputatie kunnen aanzienlijk bijdragen aan het succes en de winstgevendheid van een ticketbedrijf.

Het imago van Ticketmaster wordt beïnvloed door de manier waarop ze omgaan met de hierboven besproken factoren in hun prijsstrategie. Klanttevredenheid is van cruciaal belang, aangezien tevreden klanten meer geneigd zijn om herhaalaankopen te doen en positieve mond-tot-mondreclame te genereren. Het aanbieden van eerlijke prijzen, duidelijke kostenstructuur en transparante beleidslijnen draagt bij aan een positief imago.

De merkreputatie van Ticketmaster wordt eveneens gevormd door de manier waarop ze omgaan met externe belanghebbenden, zoals artiesten en locaties. Het nakomen van overeenkomsten en het creëren van een win-win-situatie met artiesten en locaties kan leiden tot een sterke merkreputatie en kan de toegang tot exclusieve evenementen vergemakkelijken.

Bovendien moeten bedrijven zoals Ticketmaster rekening houden met de perceptie van het publiek en de media. Negatieve publiciteit met betrekking tot prijzen, transparantie of klantenservice kan schadelijk zijn voor het imago en de merkreputatie van het bedrijf. Het is van cruciaal belang om proactief te reageren op eventuele problemen en klachten om het vertrouwen van klanten en belanghebbenden te behouden.

Kortom, in de zoektocht naar winstmaximalisatie in de evenementenbranche is het essentieel om een positief imago en een sterke merkreputatie te cultiveren en te onderhouden. Dit zal niet alleen bijdragen aan de financiële prestaties van het bedrijf, maar ook aan het behoud van klanten en zakelijke partners.

\subsection{Appendix}
Gebruik van genAI om de tekst te schrijven voor de hoofdstukken, waarbij ik een deel van een artikel bij heb geplaats om als bron van informatie te gebruiken. Nl. https://ieeexplore.ieee.org/abstract/document/7907509, SECTION IV.Discussion.
Ook het bron https://www.kennesaw.edu/coles/centers/markets-economic-opportunity/docs/walton-final.pdf, vanaf pagina 7, de vijf factoren die ticketmaster gebruikt om prijzen te bepalen.



\newpage
% De verschillende secties moeten mooi overgaan in elkaar.
% Hiervoor gebruik je bindteksten. Deze tekstjes zijn typisch de eerste
% paragrafen van de verschillende (sub)secties en leggen uit wat de
% bedoeling van de sectie is, en eventueel hoe dat relateert aan
% de vorige sectie of hoe het past in het grotere geheel.

\section{Belangen van ticketsysteembedrijven}



\subsection{Winstmaximalisatie}
Winstmaximalisatie is een cruciaal doel voor bedrijven in de ticketverkoopindustrie, en Ticketmaster is geen uitzondering. Om hun winst te maximaliseren, hanteert Ticketmaster vaak vijf belangrijke factoren bij het bepalen van de prijzen voor tickets. Hieronder worden deze factoren uitgelegd:

\vspace{5 mm}

Face Value Prijs: De face value prijs wordt beschouwd als de prijs die wordt toegepast door zowel de verkoper als de doorverkoper. Dit is een fluctuerende prijs die wordt beïnvloed door zowel vraag als aanbod. Ticketmaster maakt gebruik van dynamische prijsalgoritmen en processen om deze prijs te beïnvloeden op basis van deze factoren. Dit stelt Ticketmaster in staat om de prijs aan te passen aan de veranderende marktomstandigheden.

Service- en Orderverwerkingskosten: Ticketmaster voegt een servicekosten toe op basis van hun overeenkomst/contract met de artiest of de doorverkoper. Dit is een vaste prijs die niet verandert op basis van de nominale waarde van het werkelijke ticket zelf. Ze voegen ook een orderverwerkingskosten toe, die is ontworpen om de kosten van verzending en bescherming te dekken. De verwerkingskosten worden niet in rekening gebracht bij in-person aankopen. Eventuele resterende gelden uit de kosten worden verdeeld tussen de artiest en de verkoper.

Leveringskosten: Ticketmaster past deze prijs aan op basis van de gekozen leveringsmethode. Soms geldt deze vergoeding niet voor bepaalde aankopen. Deze vergoeding wordt extra inkomsten voor Ticketmaster.

Facilitatiekosten: De klant die zijn tickets op Ticketmaster verkoopt, beslist of deze vergoeding wordt toegepast. Deze kosten zijn bedoeld om winst terug te geven aan de daadwerkelijke locatie waar het evenement wordt gehouden. Ticketmaster ontvangt geen winst uit deze vergoeding.

Belastingen: Belastingen vormen over het algemeen geen belangrijke inkomstenbron in vergelijking met het totale inkomstenoverzicht van Ticketmaster. Deze worden bepaald door federale, staats- en lokale wetten en zijn niet onderhandelbaar bij aankoop.

\vspace{5 mm}

Het is duidelijk dat Ticketmaster een uitgebreid prijsmodel hanteert om winstmaximalisatie te bereiken. Door rekening te houden met deze factoren en hun dynamische prijsstrategie kunnen ze de prijzen aanpassen aan veranderende marktomstandigheden, waardoor ze zowel klanten als serviceproviders kunnen bedienen en tegelijkertijd streven naar winstoptimalisatie.

\newpage
\subsection{Klanttevredenheid}

Het e-ticketconcept is aantrekkelijk voor zowel klanten als dienstverleners. Vanuit het perspectief van de klant biedt het de volgende voordelen:
\vspace{5 mm}

A. Voordelen:
\begin{itemize}
    \item Snellere en handigere verificatie van een ticket;
    \item De mogelijkheid om te profiteren van een flexibel tarievenschema, met mogelijke individuele kortingen en speciale aanbiedingen;
    \item Mogelijkheid tot intrekking van verloren tickets en hun vervanging;
    \item Geen noodzaak om contant geld bij zich te hebben, bijvoorbeeld voor een plaatselijke kaartjesautomaat. Dit is vooral handig voor klanten die de vervoersdienst slechts af en toe gebruiken of zich in een andere stad bevinden.
\end{itemize}
\vspace{5 mm}

B. Nadelen:
Ondanks deze voordelen brengt het concept van e-ticketing ook enkele zorgen met zich mee, met name op het gebied van privacy. Deze zorgen omvatten:
\begin{itemize}
    \item Alomtegenwoordige klantidentificatie;
    \item De mogelijkheid van klantprofilering, zoals het creëren van bewegingspatronen;
    \item Potentiële schending van de privacy door verhoogd toezicht, wat vaak wordt aangeduid als "Het Grote Broer"-probleem.
\end{itemize}

\vspace{5 mm}
Kortom, terwijl e-ticketing aanzienlijke voordelen biedt aan klanten, zijn er ook legitieme privacykwesties die moeten worden overwogen en aangepakt in de implementatie van dit systeem. Het handhaven van een balans tussen klanttevredenheid en privacybescherming is van cruciaal belang voor het succes van e-ticketingdiensten.

\subsection{Imago en merkreputatie}

In het streven naar winstmaximalisatie speelt het imago en de merkreputatie een essentiële rol voor bedrijven in de evenementenbranche, zoals Ticketmaster. Een positief imago en een sterke merkreputatie kunnen aanzienlijk bijdragen aan het succes en de winstgevendheid van een ticketbedrijf.

Het imago van Ticketmaster wordt beïnvloed door de manier waarop ze omgaan met de hierboven besproken factoren in hun prijsstrategie. Klanttevredenheid is van cruciaal belang, aangezien tevreden klanten meer geneigd zijn om herhaalaankopen te doen en positieve mond-tot-mondreclame te genereren. Het aanbieden van eerlijke prijzen, duidelijke kostenstructuur en transparante beleidslijnen draagt bij aan een positief imago.

De merkreputatie van Ticketmaster wordt eveneens gevormd door de manier waarop ze omgaan met externe belanghebbenden, zoals artiesten en locaties. Het nakomen van overeenkomsten en het creëren van een win-win-situatie met artiesten en locaties kan leiden tot een sterke merkreputatie en kan de toegang tot exclusieve evenementen vergemakkelijken.

Bovendien moeten bedrijven zoals Ticketmaster rekening houden met de perceptie van het publiek en de media. Negatieve publiciteit met betrekking tot prijzen, transparantie of klantenservice kan schadelijk zijn voor het imago en de merkreputatie van het bedrijf. Het is van cruciaal belang om proactief te reageren op eventuele problemen en klachten om het vertrouwen van klanten en belanghebbenden te behouden.

\vspace{5 mm}
Kortom, in de zoektocht naar winstmaximalisatie in de evenementenbranche is het essentieel om een positief imago en een sterke merkreputatie te cultiveren en te onderhouden. Dit zal niet alleen bijdragen aan de financiële prestaties van het bedrijf, maar ook aan het behoud van klanten en zakelijke partners.

\subsection{Appendix}
Gebruik van genAI om de tekst te schrijven voor de hoofdstukken, waarbij ik een deel van een artikel bij heb geplaats om als bron van informatie te gebruiken. 

nl. https://ieeexplore.ieee.org/abstract/document/7907509, SECTION IV.Discussion.
Ook het bron https://www.kennesaw.edu/coles/centers/markets-economic-opportunity/docs/walton-final.pdf, vanaf pagina 7, de vijf factoren die ticketmaster gebruikt om prijzen te bepalen.

\newpage
% De verschillende secties moeten mooi overgaan in elkaar.
% Hiervoor gebruik je bindteksten. Deze tekstjes zijn typisch de eerste
% paragrafen van de verschillende (sub)secties en leggen uit wat de
% bedoeling van de sectie is, en eventueel hoe dat relateert aan
% de vorige sectie of hoe het past in het grotere geheel.

\section{Belangen van klanten/gebruikers}
\newpage
% De verschillende secties moeten mooi overgaan in elkaar.
% Hiervoor gebruik je bindteksten. Deze tekstjes zijn typisch de eerste
% paragrafen van de verschillende (sub)secties en leggen uit wat de
% bedoeling van de sectie is, en eventueel hoe dat relateert aan
% de vorige sectie of hoe het past in het grotere geheel.

\section{Machtsmisbruik door grote spelers}
\newpage
% De verschillende secties moeten mooi overgaan in elkaar.
% Hiervoor gebruik je bindteksten. Deze tekstjes zijn typisch de eerste
% paragrafen van de verschillende (sub)secties en leggen uit wat de
% bedoeling van de sectie is, en eventueel hoe dat relateert aan
% de vorige sectie of hoe het past in het grotere geheel.
\newpage
% Logischerwijze volgt er dan op het einde
% van het artikel een conclusie, waar je een bijzonder
% beknopte samenvatting (hooguit twee zinnen)
% geeft en waarin je dan uiteindelijk een mening of een uitkomst formuleert.
\newpage
% Voeg helemaal achteraan jullie artikel (zelfs achter de lijst van referenties)
% een appendix toe waarin jullie ondubbelzinnig toelichten hoe GAI is gebruikt
% in de totstandkoming van het artikel. Indien jullie geen gebruik hebben
% gemaakt van GAI, vermeld dit dan simpelweg kort in deze appendix.
% Indien jullie gebruik hebben gemaakt van GAI, maar dit niet hebben
% toegelicht in de appendix, dan zal dit beschouwd worden als plagiaat

\section{Appendix: toelichting gebruik GAI}

% TODO: opzoeken of Appendix werkt met
% specifieke LaTeX-commando's
\newpage

% Dit moet verwijderd worden en verplaats worden naar de aparte besetanden.
% Door het comment-commando uit verbatim-package verschijnt dit niet
% meer in de gecompileerde tekst.

\begin{comment}

\section{Problemen met ticketsystemen}
% ... < Text >
\newpage

\section{Capaciteitsproblemen in de ticketverkoop}
% ... < Text >
\subsection{Beschrijving capaciteitsproblemen}
\subsection{Oorzaken van capaciteitsproblemen}
\subsection{Technologische oplossingen}
\subsection{Operationele oplossingen}

\newpage
\section{Belangen van ticketsysteembedrijven}

\subsection{Winstmaximalisatie}
Winstmaximalisatie is een cruciaal doel voor bedrijven in de ticketverkoopindustrie, en Ticketmaster is geen uitzondering. Om hun winst te maximaliseren, hanteert Ticketmaster vaak vijf belangrijke factoren bij het bepalen van de prijzen voor tickets. Hieronder worden deze factoren uitgelegd:

\vspace{5 mm}

Face Value Prijs: De face value prijs wordt beschouwd als de prijs die wordt toegepast door zowel de verkoper als de doorverkoper. Dit is een fluctuerende prijs die wordt beïnvloed door zowel vraag als aanbod. Ticketmaster maakt gebruik van dynamische prijsalgoritmen en processen om deze prijs te beïnvloeden op basis van deze factoren. Dit stelt Ticketmaster in staat om de prijs aan te passen aan de veranderende marktomstandigheden.

Service- en Orderverwerkingskosten: Ticketmaster voegt een servicekosten toe op basis van hun overeenkomst/contract met de artiest of de doorverkoper. Dit is een vaste prijs die niet verandert op basis van de nominale waarde van het werkelijke ticket zelf. Ze voegen ook een orderverwerkingskosten toe, die is ontworpen om de kosten van verzending en bescherming te dekken. De verwerkingskosten worden niet in rekening gebracht bij in-person aankopen. Eventuele resterende gelden uit de kosten worden verdeeld tussen de artiest en de verkoper.

Leveringskosten: Ticketmaster past deze prijs aan op basis van de gekozen leveringsmethode. Soms geldt deze vergoeding niet voor bepaalde aankopen. Deze vergoeding wordt extra inkomsten voor Ticketmaster.

Facilitatiekosten: De klant die zijn tickets op Ticketmaster verkoopt, beslist of deze vergoeding wordt toegepast. Deze kosten zijn bedoeld om winst terug te geven aan de daadwerkelijke locatie waar het evenement wordt gehouden. Ticketmaster ontvangt geen winst uit deze vergoeding.

Belastingen: Belastingen vormen over het algemeen geen belangrijke inkomstenbron in vergelijking met het totale inkomstenoverzicht van Ticketmaster. Deze worden bepaald door federale, staats- en lokale wetten en zijn niet onderhandelbaar bij aankoop.

\vspace{10 mm}

Het is duidelijk dat Ticketmaster een uitgebreid prijsmodel hanteert om winstmaximalisatie te bereiken. Door rekening te houden met deze factoren en hun dynamische prijsstrategie kunnen ze de prijzen aanpassen aan veranderende marktomstandigheden, waardoor ze zowel klanten als serviceproviders kunnen bedienen en tegelijkertijd streven naar winstoptimalisatie.

\subsection{Klanttevredenheid}
\subsection{Imago en merkreputatie}

\section{Belangen voor klanten/gebruikers}
% ... < Text >
\subsection{Voordelen}
\subsection{Nadelen}

\section{Machtsmisbruik door grote spelers}
% ... < Text >
\subsection{Prijsstelling en commissies}
\subsection{Exclusieve deals}
\subsection{Beperking van keuze voor consumenten}
\subsection{Data verzameling en privacy}

\section{Maatregelen tegen machtsmisbruik}
% ... < Text >
\subsection{Regulering en antitrustwetten}
\subsection{Transparantie-eisen}
\subsection{Alternatieve ticketverkoopkanalen}
\subsection{Bewustwording en protest van consumenten}

\section{Conclusie}
% ... < Text >
\subsection{Samenvatting van bevindingen}
\subsection{Toekomstige ontwikkelingen in de ticketverkoopindustrie}
\subsection{Aanbevelingen voor verbeteringen in de sector}

\end{comment}

% Bibliografie.
\bibliographystyle{plain}
\bibliography{referenties.bib}
\end{document}
