\documentclass[a4paper,10pt]{article}

\usepackage[dutch]{babel}
\usepackage{graphicx}
\usepackage[colorlinks, linkcolor=black, citecolor=black, urlcolor=black]{hyperref}
\usepackage{geometry}
\geometry{tmargin=3cm, bmargin=2.2cm, lmargin=2.2cm, rmargin=2cm}
\usepackage{todonotes} %Used for the figure placeholders
\usepackage{ifthen}

\begin{document}
\newboolean{anonymize}
% Uncomment to create an anonymized version of your report
%\setboolean{anonymize}{true}

\begin{titlepage}
    \newpage
    \thispagestyle{empty}
    \frenchspacing
    \hspace{-0.2cm}
    \includegraphics[height=3.4cm]{img/kul-logo.png}
    \hspace{0.2cm}
    \rule{0.5pt}{3.4cm}
    \hspace{0.2cm}
    \begin{minipage}[b]{8cm}
        \Large{Katholieke\newline Universiteit\newline Leuven}\smallskip\newline
        \large{}\smallskip\newline
        \textbf{Department of\newline Computer Science}\smallskip
    \end{minipage}
    \hspace{\stretch{1}}
    \vspace*{3.2cm}\vfill
    \begin{center}
        \begin{minipage}[t]{\textwidth}
            \begin{center}
                \LARGE{\rm{\textbf{\uppercase{Capaciteitsproblemen bij online ticketverkoop}}}}\\
                \Large{\rm{Oorzaken, gevolgen en oplossingen}}
            \end{center}
        \end{minipage}
    \end{center}
    \vfill
    \hfill\makebox[8.5cm][l]{
        \vbox to 7cm{\vfill\noindent
            \ifthenelse{\boolean{anonymize}}{
                {\rm \textbf{Anonymized}}\\
                {\rm Academiejaar 2023-2024}
            }{
                {\rm \textbf{Groep 2}}\\
                {\rm {Lode Dockx}}\\
                {\rm {Florian Braun}}\\
                {\rm {Thijs Creemers}}\\
                {\rm {John Cai}}\\
                {\rm {Sonia Amiri}}\\
                {\rm {Rami Berro}}\\
                {\rm {Tom Cottem}}\\[2mm]
                {\rm Academiejaar 2023-2024}

            }
        }
    }
\end{titlepage}

\tableofcontents
\newpage

\section{Problemen met ticketsystemen}
% ... < Text >
\newpage

\section{Capaciteitsproblemen in de ticketverkoop}
% ... < Text >
\subsection{Beschrijving capaciteitsproblemen}
\subsection{Oorzaken van capaciteitsproblemen}
\subsection{Technologische oplossingen}
\subsection{Operationele oplossingen}

\newpage
\section{Belangen van ticketsysteembedrijven}

\subsection{Winstmaximalisatie}
Winstmaximalisatie is een cruciaal doel voor bedrijven in de ticketverkoopindustrie, en Ticketmaster is geen uitzondering. Om hun winst te maximaliseren, hanteert Ticketmaster vaak vijf belangrijke factoren bij het bepalen van de prijzen voor tickets. Hieronder worden deze factoren uitgelegd:

\vspace{5 mm}

Face Value Prijs: De face value prijs wordt beschouwd als de prijs die wordt toegepast door zowel de verkoper als de doorverkoper. Dit is een fluctuerende prijs die wordt beïnvloed door zowel vraag als aanbod. Ticketmaster maakt gebruik van dynamische prijsalgoritmen en processen om deze prijs te beïnvloeden op basis van deze factoren. Dit stelt Ticketmaster in staat om de prijs aan te passen aan de veranderende marktomstandigheden.

Service- en Orderverwerkingskosten: Ticketmaster voegt een servicekosten toe op basis van hun overeenkomst/contract met de artiest of de doorverkoper. Dit is een vaste prijs die niet verandert op basis van de nominale waarde van het werkelijke ticket zelf. Ze voegen ook een orderverwerkingskosten toe, die is ontworpen om de kosten van verzending en bescherming te dekken. De verwerkingskosten worden niet in rekening gebracht bij in-person aankopen. Eventuele resterende gelden uit de kosten worden verdeeld tussen de artiest en de verkoper.

Leveringskosten: Ticketmaster past deze prijs aan op basis van de gekozen leveringsmethode. Soms geldt deze vergoeding niet voor bepaalde aankopen. Deze vergoeding wordt extra inkomsten voor Ticketmaster.

Facilitatiekosten: De klant die zijn tickets op Ticketmaster verkoopt, beslist of deze vergoeding wordt toegepast. Deze kosten zijn bedoeld om winst terug te geven aan de daadwerkelijke locatie waar het evenement wordt gehouden. Ticketmaster ontvangt geen winst uit deze vergoeding.

Belastingen: Belastingen vormen over het algemeen geen belangrijke inkomstenbron in vergelijking met het totale inkomstenoverzicht van Ticketmaster. Deze worden bepaald door federale, staats- en lokale wetten en zijn niet onderhandelbaar bij aankoop.

\vspace{10 mm}

Het is duidelijk dat Ticketmaster een uitgebreid prijsmodel hanteert om winstmaximalisatie te bereiken. Door rekening te houden met deze factoren en hun dynamische prijsstrategie kunnen ze de prijzen aanpassen aan veranderende marktomstandigheden, waardoor ze zowel klanten als serviceproviders kunnen bedienen en tegelijkertijd streven naar winstoptimalisatie.

\subsection{Klanttevredenheid}
\subsection{Imago en merkreputatie}

\section{Belangen voor klanten/gebruikers}
% ... < Text >
\subsection{Voordelen}
\subsection{Nadelen}

\section{Machtsmisbruik door grote spelers}
% ... < Text >
\subsection{Prijsstelling en commissies}
\subsection{Exclusieve deals}
\subsection{Beperking van keuze voor consumenten}
\subsection{Data verzameling en privacy}

\section{Maatregelen tegen machtsmisbruik}
% ... < Text >
\subsection{Regulering en antitrustwetten}
\subsection{Transparantie-eisen}
\subsection{Alternatieve ticketverkoopkanalen}
\subsection{Bewustwording en protest van consumenten}

\section{Conclusie}
% ... < Text >
\subsection{Samenvatting van bevindingen}
\subsection{Toekomstige ontwikkelingen in de ticketverkoopindustrie}
\subsection{Aanbevelingen voor verbeteringen in de sector}

% \input{tex file here}
% \newpage


\bibliographystyle{plain}
\bibliography{referenties.bib}
\end{document}