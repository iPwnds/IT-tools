% Variabelen gebruikt in template.
\newcommand{\titletext}{Capaciteitsproblemen bij online ticketverkoop}
\newcommand{\subtitletext}{Oorzaken, gevolgen en oplossingen}
\newcommand{\logopath}{img/kul-logo.png}
\newcommand{\groupmembera}{Sonia Amiri}
\newcommand{\groupmemberb}{Rami Berro}
\newcommand{\groupmemberc}{Florian Braùn}
\newcommand{\groupmemberd}{John Cai}
\newcommand{\groupmembere}{Tom Cottem}
\newcommand{\groupmemberf}{Thijs Creemers}
\newcommand{\groupmemberg}{Lode Dockx}
\newcommand{\groupnumber}{2}
\newcommand{\academicyear}{2023-2024}

% Artikel gebaseerd op LaTeX-template KUL (gevonden op Overleaf).
\documentclass[a4paper, 12pt]{article}

% Packages gebruikt door LaTeX-template KUL.
\usepackage[dutch]{babel}
\usepackage{graphicx}
\usepackage[colorlinks, linkcolor=black, citecolor=black, urlcolor=black]{hyperref}
\usepackage{geometry}
\geometry{tmargin=3cm, bmargin=2.2cm, lmargin=2.2cm, rmargin=2cm}
% Change marginparwidth to prevent issues with todonotes.
\setlength{\marginparwidth}{2cm}
\usepackage{todonotes}
\usepackage{ifthen}

% Zelf toegevoegde packages.
\usepackage{verbatim} % Voor gebruik van multi-line comments.

\begin{document}

% Voor geanonimiseerd rapport te maken (onderdeel template, irrelevant).
\newboolean{anonymize}
% Onderstaande niet uncommenten, anders geanonimiseerd rapport.
% \setboolean{anonymize}{true}

% Template titlepage.
\begin{titlepage}
    \newpage
    \thispagestyle{empty}
    \frenchspacing
    \hspace{-0.2cm}
    \includegraphics[height=3.4cm]{\logopath}
    \hspace{0.2cm}
    \rule{0.5pt}{3.4cm}
    \hspace{0.2cm}
    \begin{minipage}[b]{8cm}
        \Large{Katholieke\newline Universiteit\newline Leuven}\smallskip\newline
        \large{}\smallskip\newline
        \textbf{Departement\newline Computerwetenschappen}\smallskip
    \end{minipage}
    \hspace{\stretch{1}}
    \vspace*{3.2cm}\vfill
    \begin{center}
        \begin{minipage}[t]{\textwidth}
            \begin{center}
                \LARGE{\rm{\textbf{\uppercase{\titletext}}}}\\
                \Large{\rm{\subtitletext}}
            \end{center}
        \end{minipage}
    \end{center}
    \vfill
    \hfill\makebox[8.5cm][l]{
        \vbox to 7cm{\vfill\noindent
            \ifthenelse{\boolean{anonymize}}{
                {\rm \textbf{Anonymized}}\\
                {\rm \academicyear}
            }{
                {\rm \textbf{Groep \groupnumber}}\\
                {\rm {\groupmembera}}\\
                {\rm {\groupmemberb}}\\
                {\rm {\groupmemberc}}\\
                {\rm {\groupmemberd}}\\
                {\rm {\groupmembere}}\\
                {\rm {\groupmemberf}}\\
                {\rm {\groupmemberg}}\\[2mm]
                {\rm \academicyear}

            }
        }
    }
\end{titlepage}

% Inhoudstafel.
\tableofcontents
\newpage

% Invoeging secties uit aparte bestanden.
% Hierin vertel je waarover het artikel gaat en tracht je de interesse van de
% lezer te wekken. Het einde van de inleiding
% geeft een kort overzicht van hoe het artikel gestructureerd is. 

\newpage
% De verschillende secties moeten mooi overgaan in elkaar.
% Hiervoor gebruik je bindteksten. Deze tekstjes zijn typisch de eerste
% paragrafen van de verschillende (sub)secties en leggen uit wat de
% bedoeling van de sectie is, en eventueel hoe dat relateert aan
% de vorige sectie of hoe het past in het grotere geheel.

\section{Ticketsystemen}

\subsection{Wat is een ticketsysteem?}
 Een ticketsysteem is een geavanceerde softwaretoepassing of mechanisme dat een essentiële rol speelt bij het stroomlijnen van het beheer en de tracking van verschillende soorten verzoeken, incidenten of problemen. Deze verzoeken, vaak aangeduid als "tickets", kunnen een breed scala van categorieën omvatten, zoals klantvragen, technische ondersteuningsverzoeken, onderhoudstaken, serviceaanvragen en meer. In de kern fungeert een ticketsysteem als een gecentraliseerde hub waar individuen of gebruikers hun verzoeken indienen, en organisaties deze efficiënt kunnen afhandelen, prioriteren en oplossen.

Ticketsystemen zijn ontworpen om orde en structuur te brengen in het vaak complexe proces van het beheer van inkomende verzoeken. Ze bieden een gestructureerd kader voor het maken, toewijzen, volgen en oplossen van tickets, waardoor organisaties tijdig en efficiënte diensten of oplossingen kunnen bieden aan degenen die ze nodig hebben.

\subsection{Hoe werkt een ticketsysteem?}
 Een ticketsysteem werkt volgens een reeks goed gedefinieerde principes en processen, met de volgende belangrijke componenten en functionaliteit:
 
 \begin{description}
  \item[Ticketcreatie:] Wanneer een gebruiker of klant een probleem tegenkomt of een verzoek heeft, gebruiken ze meestal het ticketsysteem om een nieuw ticket te maken. Dit houdt in dat ze details verstrekken over het probleem of verzoek, zoals een beschrijving, relevante documenten en andere relevante informatie.

  \item[Toewijzing van Tickets:] De volgende stap van het systeem omvat het toewijzen van het ticket aan het juiste personeel of team dat verantwoordelijk is voor de afhandeling van het verzoek. Toewijzingen kunnen gebaseerd zijn op factoren zoals de aard van het ticket, de prioriteit ervan, of de beschikbaarheid en expertise van de toegewezen teamleden.

  \item[Tracking van Tickets:] Zodra toegewezen, volgt het systeem voortdurend de status en voortgang van elk ticket. Dit stelt zowel gebruikers als ondersteuningsteams in staat om de reis van het ticket van indiening tot oplossing te volgen. Transparantie en verantwoordelijkheid zijn essentiële aspecten van dit volgproces.

  \item[Communicatie en Samenwerking:] Ticketsystemen bevatten vaak ingebouwde communicatie- en samenwerkingstools. Deze functies stellen teamleden in staat om te discussiëren en informatie te delen met betrekking tot het ticket, zodat iedereen die betrokken is, goed geïnformeerd is en effectief kan samenwerken.

  \item[Prioriteit en Escalatie:] Tickets worden meestal gecategoriseerd en geprioriteerd op basis van hun urgentie en impact. Hoog-prioriteits-tickets kunnen sneller worden afgehandeld, terwijl complexe problemen mogelijk moeten worden geëscaleerd naar hogere niveaus van expertise of management voor oplossing.

  \item[Oplossing en Afsluiting:] Het uiteindelijke doel van een ticketsysteem is het vergemakkelijken van de oplossing van problemen of de uitvoering van verzoeken. Zodra een ticket is afgehandeld en het probleem is opgelost of de dienst is verleend, wordt het ticket als afgesloten gemarkeerd. Dit geeft de succesvolle voltooiing van het verzoek aan.
 \end{description}


\subsection{Waar kan een ticketsysteem gebruikt worden?}
 Ticketsystemen zijn opmerkelijk veelzijdig en vinden toepassingen in verschillende contexten en industrieën, waaronder:
 \begin{description}
   \item[Klantondersteuning:] In de zakelijke wereld vormen ticketsystemen een essentieel onderdeel van klantondersteuning. Ze dienen als het belangrijkste kanaal voor klanten om problemen te melden, hulp te vragen of vragen te stellen. Bedrijven gebruiken ticketsystemen om klantbehoeften te beheren, prioriteren en efficiënt aan te pakken, wat de klanttevredenheid verbetert.
   \item[IT Helpdesk:] Binnen organisaties vertrouwen IT-afdelingen vaak op ticketsystemen voor het beheer van IT-gerelateerde problemen en serviceverzoeken. Dit omvat het oplossen van technische problemen, het vervullen van verzoeken voor software of hardware, en het onderhouden van de digitale infrastructuur van de organisatie.
   \item[Onderhoud en Reparaties:] In industriële omgevingen worden ticketsystemen gebruikt om onderhouds- of reparatieverzoeken bij te houden en te beheren. Hierdoor wordt gegarandeerd dat apparatuur, machines of faciliteiten op de juiste manier worden onderhouden, wat de uitvaltijd minimaliseert en de operationele efficiëntie waarborgt.
   \item[Projectmanagement:] Sommige projectmanagement- en taakvolgsystemen hanteren ticketgebaseerde benaderingen om projecten en opdrachten te beheren. Deze methodologie maakt georganiseerd volgen van taken, hun status en voltooiing mogelijk.
   \item[Evenementenbeheer:] Evenementplanners en -organisatoren gebruiken ticketsystemen om logistiek te beheren en problemen met betrekking tot evenementen op te lossen. Dit kan toegangsbewijzen betreffen, het volgen van verzoeken met betrekking tot het evenement en het coördineren van de inspanningen van diverse teams.
  \end{description}
- Onderwijs: In onderwijsinstellingen worden ticketsystemen gebruikt voor een breed scala van administratieve en academische doeleinden. Ze kunnen worden ingezet voor het afhandelen van studentenvragen, het beheren van facilitair onderhoud en het volgen van verzoeken met betrekking tot academische ondersteuning.

\newpage
% De verschillende secties moeten mooi overgaan in elkaar.
% Hiervoor gebruik je bindteksten. Deze tekstjes zijn typisch de eerste
% paragrafen van de verschillende (sub)secties en leggen uit wat de
% bedoeling van de sectie is, en eventueel hoe dat relateert aan
% de vorige sectie of hoe het past in het grotere geheel.

\section{Casus: Taylor Swift}
Onlangs werd een piek in anti-Ticketmaster-sentimenten veroorzaakt door de verkoop van de superster Taylor Swift's Eras Tour in november 2022. De Eras-tournee werd voorspeld als een van de best verkopende tours in de geschiedenis, aangezien dit Swift's comeback was na de Reputation-wereldtournee van 2018, de op twee na best verkopende vrouwelijke tournee in de geschiedenis. Ticketmaster meldde dat meer dan 14 miljoen gebruikers op hun site verschenen op de dag van de voorverkoop, ondanks dat ze beweerden slechts 1,5 miljoen toegangscodes aan fans te hebben vrijgegeven. Fans waren woedend over het lange wachten, het haperende systeem, het gebrek aan tickets en de onmiddellijke stijging van de kosten.

ook veel politieke backlash:
De Amerikaanse vertegenwoordiger Alexandria Ocasio-Cortez was een van de velen die de tegenreactie tegen Ticketmaster aanmoedigde na het fiasco met de verkoop van Taylor Swift-tickets. Ze herinnerde haar fans op Twitter eraan dat "Ticketmaster een monopolie is, de fusie met LiveNation nooit had mogen worden goedgekeurd, en ze moeten worden beteugeld. Splits ze op."

De belangrijkste reden voor de tegenreactie was de sterke prijsstijgingen (van \$900 naar tienduizenden dollars) en de lange wachtrijen.

\newpage
% De verschillende secties moeten mooi overgaan in elkaar.
% Hiervoor gebruik je bindteksten. Deze tekstjes zijn typisch de eerste
% paragrafen van de verschillende (sub)secties en leggen uit wat de
% bedoeling van de sectie is, en eventueel hoe dat relateert aan
% de vorige sectie of hoe het past in het grotere geheel.

\section{Capaciteitsproblemen bij overbevraging online diensten}
\newpage
\input{onderdelen/4_belangen-bedrijven_john.tex}
\newpage
% De verschillende secties moeten mooi overgaan in elkaar.
% Hiervoor gebruik je bindteksten. Deze tekstjes zijn typisch de eerste
% paragrafen van de verschillende (sub)secties en leggen uit wat de
% bedoeling van de sectie is, en eventueel hoe dat relateert aan
% de vorige sectie of hoe het past in het grotere geheel.

\section{Belangen van klanten/gebruikers}
Ticketsystemen hebben ten opzichte van de gebruiker een hele boel voordelen en nadelen. Dit komt ook mede doordat er veel factoren komen kijken bij het maken van ticketsystemen. 
Deze voor- en nadelen kunnen de belangen van een klant toe- en tegenwerken. Een aantal van deze voor en nadelen worden beknopt besproken. 

\subsection{Voordelen}
De meeste voordelen van zulke ticketsystemen zijn redelijk voor de hand liggend, bv. minder gebruik van papier door deze online tickets.
Maar sommige voordelen maken ticketsystemen nog interesanter voor de gebruiker, zoals de gebruiksvriendelijkheid die deze systemen met zich meebrengen.
Makers van ticketsystemen designen deze zodanig dat het systeem gemakkelijk te gebruiken is door klanten en dat gebruikers zich gemakkelijk kunnen navigeren.
Ze designen het ook zodanig dat de gebruiker zo efficiënt en gemakkelijk mogelijk hun ticket kunnen aankopen. Tickets verschijnen meestal ook direct na aankop 
in de mailbox van klanten zodat ze zeker zijn van hun aankoop, als dit niet zo verloopt is er nog altijd een helpdesk aanwezig bij de meeste ticketsystemen. 
Gebruikers kunnen altijd (meestal 24/7) terecht bij zo een helpdesk, dus er is hulp ter beschikking wanneer er dan toch iets zou foutlopen.
Eens dat de gebruiker een ticket heeft gekocht kan hij/zij ook heel gemakelijk zien wat de status van hun ticket en aan de hand van real-time updates kijken of er iets niet klopt bij de aankoop.
Nog een groot voordeel van ticketsystemen voor de gebruiker is dat de ticketsystemen altijd en overal beschikbaar zijn voor de klant,
online ticketsystemen zijn altijd toegankelijk zolang de gebruiker een verbinding met het internet heeft natuurlijk. 
Klanten kunnen zelfs nog last minut tickets kopen als deze nog beschikbaar zijn.

\subsection{Nadelen}
Natuurlijk komen er bij ticketsystemen ook heel wat nadelen kijken die de belangen van een gebruiker kunnen verhinderen. Doordat deze ticketsystemen meestal online zijn is het dus nodig om een stabiele internetverbinding te hebben.
Sommige gebruikers hebben dit misschien niet waardoor er heel wat kan fout gaan door een verbinding die de hele tijd wegvalt tijdens het aankopen van tickets. Ookal worden ticketsystemen gedesigned voor een gemakkleijk gebruik,
hebben gebruikers toch een kleine technische aanleg nodig, online systemen gebruiken is minder voor de hand liggend voor bv. oudere mensen die nog niet echt bekend zijn met het internet.
Omdat deze ticketsystemen meestal via website zijn kunnen deze bij een hoog aantal gebruikers crashen, waardoor al de gebruikers het ticketsysteem niet kunnen gebruiken. Dit geldt niet voor alle ticketsystemen, de meeste hebben al grote servers de veel website bezoekers aankunnen.
Een nadeel dat tijdens het betalen van de tickets zich kan voordoen, is dat er een hoop fees en/of commissies kunnen worden opgelegt door de ticketing maatschapij voor sercvice etc., waardoor de prijs wel eens kan oplopen. 
Gebruikers kunnen ook ondervinden dat er soms maar een aantal beperkte betaalmethodes zijn die misschien niet voor de gebruiker werken. Als het betalen dan toch gelukt is kan het ook zijn dat de gebruiker niet nauwkeurig genoeg is geweest en dat de persoonlijke info op het ticket niet klopt.
Dit is iedereen wel eens overkomen maar sommige ticketsystemen hebben geen of een slechte refund policy waardoor de klant geld verliest of niet alles krijgt terugbetaald. 

\vspace{10 mm}
Ticketsystemen hebben dus zeker en vast een hoop voor en nadelen die de belangen van de gebruiker voor-of tegenwerken, maar over het algemeen zijn ze gemakkelijker en efficiënter in gebruik. 
Elke gebruiker heeft een gepersonaliseert systeem waarop ze gemakkelijk hun tickets kunnen beheren.

\newpage
% De verschillende secties moeten mooi overgaan in elkaar.
% Hiervoor gebruik je bindteksten. Deze tekstjes zijn typisch de eerste
% paragrafen van de verschillende (sub)secties en leggen uit wat de
% bedoeling van de sectie is, en eventueel hoe dat relateert aan
% de vorige sectie of hoe het past in het grotere geheel.

\section{Machtsmisbruik door grote spelers}
Machtsmisbruik door grote spelers in de ticketverkoopindustrie is een essentieel en controversieel aspect van het onderwerp. 
Het heeft verstrekkende gevolgen voor consumenten en organisatoren, en daarom verdient het onze aandacht en onderzoek.

\subsection{Prijsstelling en commissies}
De prijsstelling en commissies in de ticketverkoopindustrie roepen bezorgdheid op vanwege gebrek aan transparantie en hoge bijkomende kosten, wat zowel consumenten als organisatoren kan benadelen. 
Het vraagt om meer aandacht en regulering voor eerlijke prijzen en duidelijkheid.

\subsection{Exclusieve deals}
Exclusieve deals tussen grote ticketverkoopbedrijven en evenementorganisatoren hebben zowel voordelen als nadelen. 
Ze kunnen zorgen voor grootschalige promotie, maar beperken ook de concurrentie en kunnen leiden tot hogere ticketprijzen. 
Regulering is nodig om eerlijke concurrentie te waarborgen en machtsmisbruik te voorkomen.

\subsection{Beperking van keuze voor consumenten}
Machtsmisbruik in de ticketverkoopindustrie kan de keuzevrijheid van consumenten beperken door exclusieve deals en doorverkoopbeperkingen. 
Regulering is essentieel om de consumentenrechten te beschermen en ervoor te zorgen dat consumenten toegang hebben tot diverse verkoopopties.

\subsection{Data verzameling en privacy}
De verzameling van consumentengegevens in de ticketverkoopindustrie roept privacyzorgen op. 
Het is van essentieel belang dat ticketverkoopbedrijven de gegevens van consumenten veilig behandelen en transparant zijn over hun praktijken. 
Privacyregulering is nodig om consumenten te beschermen tegen misbruik van hun gegevens.

\vspace{10 mm}
In samenvatting, machtsmisbruik in de ticketverkoopindustrie heeft diverse vormen, waaronder exclusieve deals, prijsstelling, keuzebeperking voor consumenten en gegevensverzameling. 
Dit kan nadelige gevolgen hebben. 
Regelgeving en transparantie zijn cruciaal om eerlijke concurrentie, consumentenbescherming en integriteit in deze industrie te waarborgen. 
Een evenwichtige aanpak is essentieel voor een eerlijk en toegankelijk ticketsysteem voor alle fans.

\newpage
\input{onderdelen/7_maatregelen_rami.tex}
\newpage
% Logischerwijze volgt er dan op het einde
% van het artikel een conclusie, waar je een bijzonder
% beknopte samenvatting (hooguit twee zinnen)
% geeft en waarin je dan uiteindelijk een mening of een uitkomst formuleert.

\section{Conclusie}
\newpage
% Voeg helemaal achteraan jullie artikel (zelfs achter de lijst van referenties)
% een appendix toe waarin jullie ondubbelzinnig toelichten hoe GAI is gebruikt
% in de totstandkoming van het artikel. Indien jullie geen gebruik hebben
% gemaakt van GAI, vermeld dit dan simpelweg kort in deze appendix.
% Indien jullie gebruik hebben gemaakt van GAI, maar dit niet hebben
% toegelicht in de appendix, dan zal dit beschouwd worden als plagiaat

\newpage

% Dit moet verwijderd worden en verplaats worden naar de aparte besetanden.
% Door het comment-commando uit verbatim-package verschijnt dit niet
% meer in de gecompileerde tekst.

\begin{comment}

\section{Problemen met ticketsystemen}
% ... < Text >
\newpage

\section{Capaciteitsproblemen in de ticketverkoop}
% ... < Text >
\subsection{Beschrijving capaciteitsproblemen}
\subsection{Oorzaken van capaciteitsproblemen}
\subsection{Technologische oplossingen}
\subsection{Operationele oplossingen}

\newpage
\section{Belangen van ticketsysteembedrijven}

\subsection{Winstmaximalisatie}
Winstmaximalisatie is een cruciaal doel voor bedrijven in de ticketverkoopindustrie, en Ticketmaster is geen uitzondering. Om hun winst te maximaliseren, hanteert Ticketmaster vaak vijf belangrijke factoren bij het bepalen van de prijzen voor tickets. Hieronder worden deze factoren uitgelegd:

\vspace{5 mm}

Face Value Prijs: De face value prijs wordt beschouwd als de prijs die wordt toegepast door zowel de verkoper als de doorverkoper. Dit is een fluctuerende prijs die wordt beïnvloed door zowel vraag als aanbod. Ticketmaster maakt gebruik van dynamische prijsalgoritmen en processen om deze prijs te beïnvloeden op basis van deze factoren. Dit stelt Ticketmaster in staat om de prijs aan te passen aan de veranderende marktomstandigheden.

Service- en Orderverwerkingskosten: Ticketmaster voegt een servicekosten toe op basis van hun overeenkomst/contract met de artiest of de doorverkoper. Dit is een vaste prijs die niet verandert op basis van de nominale waarde van het werkelijke ticket zelf. Ze voegen ook een orderverwerkingskosten toe, die is ontworpen om de kosten van verzending en bescherming te dekken. De verwerkingskosten worden niet in rekening gebracht bij in-person aankopen. Eventuele resterende gelden uit de kosten worden verdeeld tussen de artiest en de verkoper.

Leveringskosten: Ticketmaster past deze prijs aan op basis van de gekozen leveringsmethode. Soms geldt deze vergoeding niet voor bepaalde aankopen. Deze vergoeding wordt extra inkomsten voor Ticketmaster.

Facilitatiekosten: De klant die zijn tickets op Ticketmaster verkoopt, beslist of deze vergoeding wordt toegepast. Deze kosten zijn bedoeld om winst terug te geven aan de daadwerkelijke locatie waar het evenement wordt gehouden. Ticketmaster ontvangt geen winst uit deze vergoeding.

Belastingen: Belastingen vormen over het algemeen geen belangrijke inkomstenbron in vergelijking met het totale inkomstenoverzicht van Ticketmaster. Deze worden bepaald door federale, staats- en lokale wetten en zijn niet onderhandelbaar bij aankoop.

\vspace{10 mm}

Het is duidelijk dat Ticketmaster een uitgebreid prijsmodel hanteert om winstmaximalisatie te bereiken. Door rekening te houden met deze factoren en hun dynamische prijsstrategie kunnen ze de prijzen aanpassen aan veranderende marktomstandigheden, waardoor ze zowel klanten als serviceproviders kunnen bedienen en tegelijkertijd streven naar winstoptimalisatie.

\subsection{Klanttevredenheid}
\subsection{Imago en merkreputatie}

\section{Belangen voor klanten/gebruikers}
% ... < Text >
\subsection{Voordelen}
\subsection{Nadelen}

\section{Machtsmisbruik door grote spelers}
% ... < Text >
\subsection{Prijsstelling en commissies}
\subsection{Exclusieve deals}
\subsection{Beperking van keuze voor consumenten}
\subsection{Data verzameling en privacy}

\section{Maatregelen tegen machtsmisbruik}
% ... < Text >
\subsection{Regulering en antitrustwetten}
\subsection{Transparantie-eisen}
\subsection{Alternatieve ticketverkoopkanalen}
\subsection{Bewustwording en protest van consumenten}

\section{Conclusie}
% ... < Text >
\subsection{Samenvatting van bevindingen}
\subsection{Toekomstige ontwikkelingen in de ticketverkoopindustrie}
\subsection{Aanbevelingen voor verbeteringen in de sector}

\end{comment}

% Bibliografie.
\bibliographystyle{plain}
\bibliography{referenties.bib}
\end{document}
