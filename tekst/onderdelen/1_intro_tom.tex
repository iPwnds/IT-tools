% Hierin vertel je waarover het artikel gaat en tracht je de interesse van de
% lezer te wekken. Het einde van de inleiding
% geeft een kort overzicht van hoe het artikel gestructureerd is. 

\section{Inleiding}

In 2022 ontstond ophef vanwege de chaotische online ticketverkoop voor 
Taylor Swift-concerten door Ticketmaster. Het Amerikaanse bedrijf werd
geconfronteerd met capaciteitsproblemen doordat veel fans tegelijkertijd
zijn website wilden raadplegen \cite{kelley2022senate,belga2022taylor}.
Aandacht voor dergelijke capaciteitsproblemen en mogelijke oplossingen
is belangrijk, aangezien die storingen kosten en ontevredenheid veroorzaken en
ze zich wel vaker voordoen in de sector \cite{sisario2023ticketmaster,
timsit2023its} en bij online diensten in het algemeen \cite{warren2023chatgpt, 
warren2020discord, reuters2023what}.
Daarom beantwoordt deze paper de volgende vragen:
"Wat zijn capaciteitsproblemen en hun mogelijke oplossingen?",
"Welke bedrijven zijn actief in de ticketverkoopsector en hoeveel (relatieve)
macht hebben ze?",
"Hoe worden de belangen van bedrijven en klanten
geïmpacteerd door de marktsituatie en capaciteitsproblemen?" en
"In welke mate is er economisch machtsmisbruik en welke maatregelen kunnen
dat verhinderen?".
Eerst volgt een bespreking van de ticketverkoopsector en
zijn actuele capaciteitsproblemen.
Daarna wordt algemener ingegaan op dergelijke problemen en hun oplossingen,
los van de concrete sector.
In het vervolg focust de paper opnieuw op de casus van ticketverkoop.
De belangen van bedrijven en klanten worden besproken,
waarna wordt afgesloten met een behandeling van machtsmisbruik en
maatregelen daartegen.