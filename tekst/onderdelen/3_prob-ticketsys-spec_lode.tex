% De verschillende secties moeten mooi overgaan in elkaar.
% Hiervoor gebruik je bindteksten. Deze tekstjes zijn typisch de eerste
% paragrafen van de verschillende (sub)secties en leggen uit wat de
% bedoeling van de sectie is, en eventueel hoe dat relateert aan
% de vorige sectie of hoe het past in het grotere geheel.

\section{Casus: Taylor Swift}


Onlangs werd er een piek in anti-Ticketmaster-sentimenten veroorzaakt door de verkoop van de superster Taylor Swift haar “Eras Tour” in november 2022. 
De Eras-tournee werd voorspeld als een van de bestverkopende tours in de geschiedenis, aangezien dit Swift's comeback was na haar “Reputation” wereldtour van 2018, de op twee na bestverkopende vrouwelijke tournee in de geschiedenis.
Ticketmaster meldde dat meer dan 14 miljoen gebruikers op hun site verschenen op de dag van de voorverkoop, desondanks dat ze beweerden slechts 1,5 miljoen toegangscodes aan fans te hebben vrijgegeven. 
Haar fans waren woedend over de lange wachttijden, het haperende systeem, het gebrek aan tickets en de onmiddellijke stijging van de kosten.


Dit incident werd ook besproken in de politiek, waar het veel backlash kreeg.
De Amerikaanse vertegenwoordiger Alexandria Ocasio-Cortez was een van de velen die de tegenreactie tegen Ticketmaster aanmoedigde na het fiasco met de verkoop van Taylor Swift-tickets. Ze herinnerde haar fans op Twitter eraan dat "Ticketmaster een monopolie is, de fusie met LiveNation nooit had mogen worden goedgekeurd, en ze moeten worden beteugeld. Splits ze op."

De belangrijkste reden voor de tegenreactie was de sterke prijsstijgingen (van \$900 naar tienduizenden dollars) en de lange wachtrijen.


Ticketmaster controleert meer dan 70\% van de markt wat betreft tickets verkopen en live events.
Hierdoor kunnen ze als een monopolie beschouwd worden, dat geen concurrentie heeft en daardoor zijn macht kan misbruiken. Dit werd zeer duidelijk bij de verkoop van Taylor Swift haar “Eras tour”. Zonder concurrentie konden ze zonder gevolgen de ticketprijzen drastisch doen stijgen. 

