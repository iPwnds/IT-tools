% De verschillende secties moeten mooi overgaan in elkaar.
% Hiervoor gebruik je bindteksten. Deze tekstjes zijn typisch de eerste
% paragrafen van de verschillende (sub)secties en leggen uit wat de
% bedoeling van de sectie is, en eventueel hoe dat relateert aan
% de vorige sectie of hoe het past in het grotere geheel.

\section{Casus: Taylor Swift}


Onlangs werd er een piek in anti-Ticketmaster-sentimenten veroorzaakt door de verkoop van de superster Taylor Swift haar “Eras Tour” in november 2022. 
De Eras-tournee werd voorspeld als een van de bestverkopende tours in de geschiedenis, aangezien dit Swift's comeback was na haar “Reputation” wereldtour van 2018, de op twee na bestverkopende vrouwelijke tournee in de geschiedenis.
Ticketmaster meldde dat meer dan 14 miljoen gebruikers op hun site verschenen op de dag van de voorverkoop, desondanks dat ze beweerden slechts 1,5 miljoen toegangscodes aan fans te hebben vrijgegeven. 
Haar fans waren woedend over de lange wachttijden, het haperende systeem, het gebrek aan tickets en de onmiddellijke stijging van de kosten.


Dit incident werd ook besproken in de politiek, waar het veel backlash kreeg.
De Amerikaanse vertegenwoordiger Alexandria Ocasio-Cortez was een van de velen die de tegenreactie tegen Ticketmaster aanmoedigde na het fiasco met de verkoop van Taylor Swift-tickets. Ze herinnerde haar fans op Twitter eraan dat "Ticketmaster een monopolie is, de fusie met LiveNation nooit had mogen worden goedgekeurd, en ze moeten worden beteugeld. Splits ze op."

De belangrijkste reden voor de tegenreactie was de sterke prijsstijgingen (van \$900 naar tienduizenden dollars) en de lange wachtrijen.


The term monopoly is employed when describing a company that has exclusive control of a market.
This means that there will only be a single seller for a certain type of product or service.
Businesses like these make it almost impossible for smaller companies to compete,
limiting the diversity in their respective market, and often raising the prices that consumers must pay.




Ticketmaster controleert meer dan 70\% van de markt wat betreft tickets verkopen en live events.
Hierdoor kunnen ze als een monopolie beschouwd worden, dat geen concurrentie heeft en daardoor zijn macht kan misbruiken. Dit werd zeer duidelijk bij de verkoop van Taylor Swift haar “Eras tour”. Zonder concurrentie konden ze zonder gevolgen de ticketprijzen drastisch doen stijgen. 

De fans van Swift, bekend als "Swifties", waren bijzonder boos over het fiasco, van wie sommigen juridische stappen ondernamen. Advocaten die fans waren van Swift mobiliseerden een grassrootsgroep genaamd Vigilante Legal LLC., een toneelstuk op Swift's nummer "Vigilante Shit" uit 2022; de kerngroep bestaat uit meer dan vijftig professionals met een achtergrond in de rechten, de overheid, public relations en computerwetenschappen - "iedereen, van advocaten tot mensen die in de financiële of banksector werken, tot mensen met antitrustervaring." De groep werd opgericht door de Amerikaanse advocaat en fan van Swift, Blake Barnett. Ze meldde dat ze op 18 november 2022 meer dan 1.200 reacties had ontvangen. Vigilante Legal begon ook bewijsmateriaal te verzamelen van fans die "discriminerende en twijfelachtige service" van Ticketmaster ervoeren, waaronder een mogelijke overtreding van de American with Disabilities Act van 1990. De groep heeft de klachten verzameld om voor te leggen aan de Federal Trade Commission en de procureurs-generaal in elke Amerikaanse staat.

Voters of Tomorrow, een politiek-activistische organisatie onder leiding van generatie Z, opende een antitrustinitiatief genaamd "S.W.I.F.T." (Swifties Working to Verhoog Fairness from Ticketmaster) op 17 november, met als doel "Gen Z-organisatoren samen te brengen om te pleiten voor wetgeving die de federale autoriteit uitbreidt om toekomstige monopolies rond entertainment te overzien en te voorkomen."

Om scalpers te ontwijken, gebruikten sommige fans hun ‘hechte gemeenschap’ op sociale mediaplatforms om accounts te vormen zoals ‘@ErasTourResell’ en ‘TS Tour Connect’ om een netwerk van spreadsheets, Google Forms en online bulletinboards te organiseren. die de uitwisseling van kaartjes tegen nominale kosten mogelijk maakte van fans die wilden doorverkopen aan fans die wilden kopen. Vrijwilligers van het initiatief werkten aan de inzendingen van kaartjes, verifieerden deze via schermopnames en bevestigingsmails en plaatsten de kaartjeslijsten op de post. ErasTourResell wordt geleid door drie fans van Swift, namelijk Courtney Johnston, Channette Garay en Angel Richards. Volgens The New York Times heeft ErasTourResell alleen al in maart 2023 meer dan 1.300 tickettransacties tussen fans geregeld. Volgens The Wall Street Journal heeft ErasTourResell vanaf juli 2023 meer dan 3.000 fans aan kaartjes tegen de nominale waarde geholpen.


