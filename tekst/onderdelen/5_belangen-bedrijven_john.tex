% De verschillende secties moeten mooi overgaan in elkaar.
% Hiervoor gebruik je bindteksten. Deze tekstjes zijn typisch de eerste
% paragrafen van de verschillende (sub)secties en leggen uit wat de
% bedoeling van de sectie is, en eventueel hoe dat relateert aan
% de vorige sectie of hoe het past in het grotere geheel.
\section{Belangen van ticketsysteembedrijven}

% Deze sectie gaat in op hoe de belangen van ticketsysteembedrijven worden beïnvloed
% door de huidige marktsituatie en de capaciteitsproblemen.

% Idee sectie: Het voornaamste belang is winst. Winst wordt op die manier beïnvloed door die twee zaken. Het standpunt van de eigneaars / aandeelhouders
% want dat zijn diegene waaraan bedrijf finaaal verantwoording moet afleggen. Zowel op korte termijn als op lange termijn vrijwinst belangrijk.
% Refereren naar dingen die Florian zegt in machtsmisbruik over hoe marksituatie zoals besproken door Sonia en Loda gebruikt worden om winst vergroten
% (positief voor bedrijf). Maar heeft keerzijde (zie elementen aangeheeld in imago en marktreputatie, wordt beïnvloed door problemen met capaciteit)
% bijkomend politieke vragen over het evt. opbreken kan op lange temrijn velries deel bedrijf en dus deel winst betekenen. Dus bedrijf ergens een balans zoeken.

Voordat we beginnen, bespreken we kort de huidige marktsituatie en concurrentie. Ticketmaster is tegenwoordig een van de meest prominente primaire 
ticketverkopers omdat ze in eerste instantie tickets verkopen voor evenementen en concerten, rechtstreeks aan consumenten. In tegenstelling tot 
hun concurrenten, zoals Eventbrite en StubHub \cite{Competitors:online}, die deel uitmaken van de secundaire markt, waar al verkochten tickets 
vaak tegen een hogere prijs worden doorverkocht, soms zelfs met een stijging van 273\% boven de nominale waarde van het ticket \cite{tompkinsanalysis}. 

\vspace{5mm}
Wat betreft de ticketmarkt benadrukt Live Nation, het moederbedrijf van Ticketmaster, dat concertkaartjes nog steeds 'dramatisch onder-geprijsd'
zijn volgens hun analyse van de secundaire markt voor doorverkoop \cite{Daniel:online}. Hierbij wordt opgemerkt dat gemiddelde secundaire ticketprijzen 
bijna twee keer zo hoog blijven als die van primaire tickets. CEO Michael Rapino beweert daarom dat de vraag de prijzen nog steeds rechtvaardigt.

\subsection{Winstmaximalisatie}
Ticketsysteembedrijven, waaronder Ticketmaster, Eventbrite, StubHub en anderen, opereren in een dynamische markt waar zowel de huidige 
marktsituatie als capaciteitsproblemen hun belangen beïnvloeden, zoals eerder besproken. Winstmaximalisatie staat centraal in de doelstellingen 
van deze bedrijven.

\vspace{5mm}
Hoewel het op het eerste gezicht lijkt alsof de opkomst van meerdere secundaire marktbedrijven een negatieve invloed zou hebben op 
winstmaximalisatie, is het tegendeel waar. In plaats van de secundaire markt tegen te werken, hebben veel primaire ticketverkopers ervoor gekozen 
om bedrijven in de secundaire markt over te nemen, zoals Live Nation dat TicketsNow heeft verworven \cite{Holmstrom2019}.

\vspace{5mm}
Dit leidde tot controverse, aangezien Ticketmaster beweerde maatregelen te nemen tegen "scalping", maar in feite schond het zijn eigen regels zoals 
eerder vermeld. \cite{CBC:online}

\vspace{5mm}
Ondanks het belang van winstmaximalisatie bestaat de mogelijkheid dat grote spelers zoals Ticketmaster soms misbruik maken van hun macht, met name 
op het gebied van prijsstelling, en dit onderwerp zal verder in detail worden toegelicht in Hoofdstuk \ref{sec:Hoofdstuk_7}.


\subsection{Imago en merkreputatie}

Naast streven naar winstmaximalisatie is het imago cruciaal voor bedrijven zoals Ticketmaster. De perceptie van Ticketmaster wordt beïnvloed door 
de aanpak van capaciteitsproblemen en interacties met externe belanghebbenden, zoals artiesten. Bijvoorbeeld, de verkoopchaos rond Taylor Swift's 
Eras Tour leidde tot een rechtszaak tegen Live Nation wegens vermeende monopolie vorming en schadelijke praktijken, zoals besproken in Hoofdstuk \ref{sec:Hoofdstuk_3}
\cite{Knack:online}.

\vspace{5mm}

Ticketmaster en vergelijkbare bedrijven moeten letten op publieke en mediapercepties, die niet alleen financiële prestaties, zoals de aandeelwaarde 
van Live Nation, beïnvloeden, maar ook klantenbinding en juridische kwesties. In het volgende deel behandelen we klantbelangen.

% \subsection{Appendix}
% Gebruik van genAI om de tekst te schrijven voor de hoofdstukken, waarbij ik een deel van een artikel bij heb geplaats om als bron van informatie te gebruiken. 

% nl. https://ieeexplore.ieee.org/abstract/document/7907509, SECTION IV.Discussion.
% Ook het bron https://www.kennesaw.edu/coles/centers/markets-economic-opportunity/docs/walton-final.pdf, vanaf pagina 7, de vijf factoren die ticketmaster gebruikt om prijzen te bepalen.
