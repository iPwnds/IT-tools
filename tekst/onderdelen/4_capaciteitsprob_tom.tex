% De verschillende secties moeten mooi overgaan in elkaar.
% Hiervoor gebruik je bindteksten. Deze tekstjes zijn typisch de eerste
% paragrafen van de verschillende (sub)secties en leggen uit wat de
% bedoeling van de sectie is, en eventueel hoe dat relateert aan
% de vorige sectie of hoe het past in het grotere geheel.

\section{Capaciteitsproblemen bij overbevraging online diensten}

Deze sectie gaat in op het technische aspect van de problemen bij Ticketmaster,
door capaciteitstekort en oplossingen daarvoor te bespreken. Aangezien dit
een algemene IT-problematiek is, die zich voordoet in andere bedrijven en
sectoren \cite{warren2023chatgpt, warren2020discord, reuters2023what},
wordt dit algemeen behandeld. 

Om een online dienst of website te gebruiken, moeten gebruikers de
server(s) van die dienst of website bereiken. In die context is er sprake van
capaciteitstekort wanneer de serverinfrastructuur onvoldoende systeemresources
heeft om alle inkomende verzoeken van gebruikers af te handelen.
Dat kan ertoe leiden dat systemen trager zijn (bv. lange laattijd website),
slecht functioneren (bv. slechte weergave van site/applicatie) of
helemaal niet meer werken (bv. foutmeldingen zoals HTTP-statuscode
503)\footnote{"503 Service Unavailable" \cite{fielding2022http}}
\cite{guitart2010survey, guitart2007designing}.
Het probleem kan zich continu voordoen, wanneer de infrastructuur onvoldoende
is om aan de normale vraag tegemoet te komen,
of van tijdelijke aard zijn, wanneer belastingspieken niet kunnen worden
opgevangen \cite{schroeder2006web}. Pieken kunnen veroorzaakt worden
door verschillende soorten verzoeken:
normale (bv. Black Friday-toeloop \cite{iyer2001overload}),
onwenselijke (bv. web scraping \cite{thelwall2006web}) of
malafide (bv. cyberaanval \cite{loukas2009protection}).

Verschillende technieken kunnen capaciteitsproblemen verminderen.
Hieronder worden de voornaamste vermeld (niet-exhaustief):

\begin{description}
    \item [Verticaal opschalen:] opwaarderen (beschikbare)
    systeemresources huidige server(s) of virtuele machine(s)
    (bv. geheugen, processors en bandbreedte)
    \cite{lu2014applicationdriven, michael2007scaleup, appuswamy2013scaleup}

    \item [Horizontaal opschalen:] voorzien meerdere servers of virtuele
    machines, waarover verzoeken worden verdeeld
    \cite{lu2014applicationdriven, michael2007scaleup}

    \item [Evalueren en simuleren:] (toekomstige) noden inschatten
    via prestatie-evaluaties infrastructuur, inclusief
    stresstests met gesimuleerde belastingen
    \cite{penaortiz2015generating, penaortiz2013analyzing, penaortiz2015new}

    \item [Cachen:] bijhouden (delen van) dynamisch gegenereerde wegpagina's
    of database-queryresultaten op snellere opslag (bv. RAM)
    om gebruik systeemresources te verminderen indien gebruikers
    gelijkaardige pagina's of data opvragen
    \cite{sivasubramanian2007analysis,amazon2023whatiscaching}
    
    \item [Optimaliseren algoritmen en datastructuren:] efficiënte code en
    datastructuren gebruiken minder systeemresources
    \cite{feitelson2015introduction}

    \item [Optimaliseren software-architectuur en -ontwerp:] hogere
    efficiëntie \cite{feitelson2015introduction} en
    opschalingsopportuniteiten via weldoordachte
    architecturen en ontwerpen (bv. microservices-benadering waarbij modulaire
    services onafhankelijk worden opgeschaald o.b.v. noden)
    \cite{coulson2020adaptive, thones2015microservices}

\end{description}

Specifieke implementaties van opschalen en cachen kunnen capaciteitsproblemen
extra reduceren:

\begin{description}
    \item [Geografisch distribueren:] opschalen of cachen via
    geografisch verspreidde servers voor snellere, regionalere connecties
    \cite{sivasubramanian2007analysis, chenhao2017mitigating,
    colajanni1998dynamic}
    
    \item [Optimaliseren loadbalancingalgoritmes:] vergelijking opties 
    en keuze voor dynamische algoritmes die verzoeken efficiënt over
    servers verdelen o.b.v. omstandigheden
    \cite{mourad1997scalable, zhou2023comparative, amazon2023whatisload}
    \footnote{De load balancer zit tussen gebruiker en servers als
    facilitator die verkeer verdeelt of herleidt \cite{amazon2023whatisload}.}

    \item [Automatiseren infrastructuurbeheer en goed monitoren:] vergroting
    flexibiliteit/reactiviteit door extensieve monitoring belasting,
    automatische opschaling en automatische infrastructuur-configuratie
    door gebruik 'Infrastructure as Code (IaC)'
    \footnote{IaC maakt configuraties-sjablonen zodat inzet nieuwe
    servers wordt geautomatiseerd. \cite{microsoft2023wat}}
    en hulpprogramma's \cite{microsoft2023wat}
    
    \item [Gebruiken cloudproviders:] (automatisch) gebruik flexibiliteit
    cloudinfrastructuur die snel
    servers in verschillende locaties
    kan starten, configureren en afsluiten ('Infrastructure as a Service') 
    \cite{microsoft2023wat, gandhi2018modeldriven, hwang2014scaleout,
    chenhao2017mitigating}

\end{description}





% TODO: Vermelden dat we specifiek focussen op wat bedrijven zelf kunnen doen.
% Want ook fundamenteler kan er op besturingssysteemniveau iets gedaan worden
% voor betere resource allocation:
% G. Banga, P. Druschel and J.C. Mogul, Resource containers: A new facility for resource management in server systems, in: 3rd Symposium on Operating Systems Design and Implementation, New Orleans, LA, February 1999.
% P. Druschel and Gaurav Banga, Lazy Receiver Processing (LRP): A network subsystem architecture for server systems, in: OSDI '96, Seattle, Washington, October 1996.
% J. Mogul and K.K. Ramakrishnan, Eliminating receive livelock in an interrupt-driven kernel, in: USENIX '96, San Diego, CA, January 1996.