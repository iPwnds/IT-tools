% De verschillende secties moeten mooi overgaan in elkaar.
% Hiervoor gebruik je bindteksten. Deze tekstjes zijn typisch de eerste
% paragrafen van de verschillende (sub)secties en leggen uit wat de
% bedoeling van de sectie is, en eventueel hoe dat relateert aan
% de vorige sectie of hoe het past in het grotere geheel.

\section{Capaciteitsproblemen bij overbevraging online diensten}

Deze sectie gaat in op het technische aspect van de problemen bij Ticketmaster,
door capaciteitstekort en oplossingen daarvoor te bespreken. Aangezien dit
een algemene IT-problematiek is, die zich voordoet in andere bedrijven en
sectoren, wordt dit algemeen behandeld. 

Om een online dienst of website te gebruiken, moeten gebruikers de
server(s) van die dienst of website bereiken. In die context is er sprake van
capaciteitstekort wanneer de serverinfrastructuur onvoldoende systeemresources
heeft om alle inkomende verzoeken van gebruikers af te handelen.
Dat kan ertoe leiden dat systemen trager zijn (bv. lange laattijd website),
slecht functioneren (bv. slechte weergave van site/applicatie) of
helemaal niet meer werken (bv. foutmeldingen zoals HTTP-statuscode 503
\footnote{Deze statuscode geeft aan dat de dienst niet beschikbaar is}).
Het probleem kan zich continu voordoen, wanneer de infrastrctuur onvoldoende
is om aan de normale vraag tegemoet te komen,
of van tijdelijke aard zijn, wanneer de flexibiliteit ontbreekt om
belastingspieken op te vangen.

\newlinechar

Mogelijke oplossingen voor capaciteitsproblemen zijn de volgende:

\begin{itemize}
    \item Verticaal opschalen: opwaarderen hardware huidige server(s)
    (bv. extra geheugen, betere processors of extra bandbreedte)

    \item Horizontaal opschalen: voorzien van extra servers,
    waarover de verzoeken kunnen worden verdeeld

    \item Geografisch distribueren: horizontaal opschalen
    waarbij gewerkt wordt met geografisch verspreidde servers
    zodat gebruikers dichtsbijzijnde server voor hun locatie kunnen gebruiken
    met oog op snellere connectie

    \item Kwalitatief load balancen: optimaliseren van de algoritmen die
    de verzoeken over de verschillende servers verdelen
    % Notitie: maakt gebruik van virtuele server die verzoeken verdeelt

    \item Automatiseren infrastructuurbeheer en goed monitoren: vergroten
    flexibiliteit en reactiviteit door automatiseren van opschaling en
    configuratie infrastructuur en door extensieve monitoring van belasting
    % Notitie: Staat doorgaans centraal in DevOps.
    % Zie bron Microsoft Azure.
    % Configuratiebeheer, gebruik van hulpprogramma's of IaC
    % om van configuratie sjablonen te maken zodat configuratie
    % kan worden geautomatiseerd en complexe omgeving makkelijk kan worden
    % opgeschaald (systeemresources definiëren in code om snel, herhaalbaar en
    % controleerbaar te implementeren zonder menselijke fouten. Ook idee
    % bewakingm waarbij men focust op real-time zichtbaarheid prestaties en
    % status infrastructuur (telemetrie. metadata en automatische meldingen
    % voor operatoren).
    
    \item Gebruikken cloudproviders: (automatisch) opschalen vereenvoudigen
    door gebruiken van cloudinfrastructuur.
    % Notitie: Staat doorgaans ook centraal in DevOps.
    % Zie opnieuwe bron Microsoft Azure.  
    % Cloud biedt flexibiliteit waarbij je snel omgevingen in verschillende
    % geografische locaties kunt opstarten, configureren en weer afsluiten
    % afhankelijk van noden.

    \item Prestaties evalueren en simuleren: (toekomstige) noden infrastructuur
    inschatten door evalueren prestaties infrastructuur, inclusief
    stress testen prestaties bij gesimuleerde, zelf gegenereerde
    workloads 

    \item Cachen: kan laattijden verminderen.

    \item Optimaliseren broncode: optimaliseren van de code van de software
    vermindert de systeemresources die de servers beschikbaar moeten hebben

    \item Optimaliseren software-architectuur en -ontwerp: vereenvoudigen van
    opschaling door gebruik van flexibileren architecturen en ontwerpen 
    (bv. micro-services)

\end{itemize}


% TODO: Misschien nog behandelen: welke technieken gebruikt Ticketmaster al?
% TODO: Referenties