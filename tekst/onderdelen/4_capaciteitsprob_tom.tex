% De verschillende secties moeten mooi overgaan in elkaar.
% Hiervoor gebruik je bindteksten. Deze tekstjes zijn typisch de eerste
% paragrafen van de verschillende (sub)secties en leggen uit wat de
% bedoeling van de sectie is, en eventueel hoe dat relateert aan
% de vorige sectie of hoe het past in het grotere geheel.

\section{Capaciteitsproblemen bij overbevraging}

Deze sectie gaat in op de technische aspecten,
door capaciteitstekorten en oplossingen daarvoor te bespreken.
Dit wordt algemeen behandeld,
aangezien het een IT-problematiek is 
die zich ook voordoet in andere bedrijven en
sectoren \cite{warren2023chatgpt, warren2020discord, reuters2023what}.

Om een online dienst/website te gebruiken, moeten gebruikers de
server(s) daarvan bereiken. Er is sprake van
capaciteitstekort wanneer de serverinfrastructuur onvoldoende systeemresources
heeft om alle inkomende verzoeken af te handelen.
Dat kan ertoe leiden dat systemen trager zijn (bv. laattijden),
slecht functioneren (bv. slechte weergave site/applicatie) of
niet meer werken (bv. HTTP-statuscode
503)\footnote{"Service Unavailable" \cite{fielding2022http}}
\cite{guitart2010survey, guitart2007designing}.
Het probleem kan continu zijn, wanneer normale belasting problemen
geeft, of slechts tijdelijk, wanneer belastingspieken niet kunnen worden
opgevangen \cite{schroeder2006web}. Pieken kennen verschillende oorzaken:
normale (bv. Black Friday-toeloop \cite{iyer2001overload}),
onwenselijke (bv. web scraping \cite{thelwall2006web}) en
malafide (bv. cyberaanval \cite{loukas2009protection}).

De voornaamste technieken tegen capaciteitsproblemen zijn:

\begin{description}
    \item [Verticaal opschalen:] opwaarderen (beschikbare)
    systeemresources huidige server(s) of virtuele machine(s)
    (bv. geheugen, processors en bandbreedte)
    \cite{lu2014applicationdriven, michael2007scaleup, appuswamy2013scaleup}

    \item [Horizontaal opschalen:] voorzien meerdere servers of virtuele
    machines, waarover verzoeken worden verdeeld
    \cite{lu2014applicationdriven, michael2007scaleup}

    \item [Evalueren en simuleren:] (toekomstige) noden inschatten
    via prestatie-evaluaties infrastructuur, inclusief
    stresstests met gesimuleerde belastingen
    \cite{penaortiz2015generating, penaortiz2013analyzing, penaortiz2015new}

    \item [Cachen:] bijhouden (delen van) dynamisch gegenereerde wegpagina's
    of database-queryresultaten op snellere opslag (bv. RAM)
    om gebruik systeemresources te verminderen indien gebruikers
    gelijkaardige pagina's of data opvragen
    \cite{sivasubramanian2007analysis,amazon2023whatiscaching}
    
    \item [Optimaliseren algoritmen en datastructuren:] efficiënte code en
    datastructuren gebruiken minder systeemresources
    \cite{feitelson2015introduction}

    \item [Optimaliseren software-architectuur en -ontwerp:] hogere
    efficiëntie \cite{feitelson2015introduction} en
    opschalingsopportuniteiten via weldoordachte
    architecturen en ontwerpen (bv. microservices-benadering waarbij modulaire
    services onafhankelijk worden opgeschaald o.b.v. noden)
    \cite{coulson2020adaptive, thones2015microservices}

    \item [Adapteren site-/applicatieinhoud:]
    aflevering minder (resource-intensieve) inhoud
    (bv. belangrijkste informatie, lagere resoluties)
    bij piekbelasting
    \cite{andersson2006design, abdelzaher1999web, gophstein2011trading}

    \item [Selectief degraderen:]
    degradatie resource-intensieve maar minder prioritaire software-modules
    t.v.v. essentiële dienstverlening
    \cite{wang2012towards}

    \item [Beveiligen tegen cyberaanvallen:] detectie en afweer malafide
    verzoeken \cite{loukas2009protection}
    (bv. op niveau load balancer \footnote{Zie infra voor definitie.}
    \cite{amazon2023whatisload})


\end{description}

Specifieke implementaties van opschalen en cachen kunnen capaciteitsproblemen
extra reduceren:

\begin{description}
    \item [Geografisch distribueren:] opschalen of cachen via
    geografisch verspreidde servers voor snellere, regionalere connecties
    \cite{sivasubramanian2007analysis, chenhao2017mitigating,
    colajanni1998dynamic}
    
    \item [Optimaliseren loadbalancingalgoritmes:] vergelijking opties 
    en keuze voor dynamische algoritmes die verzoeken efficiënt over
    servers verdelen o.b.v. omstandigheden
    \cite{mourad1997scalable, zhou2023comparative, amazon2023whatisload}
    \footnote{De load balancer zit tussen gebruiker en servers als
    facilitator die verkeer verdeelt of herleidt \cite{amazon2023whatisload}.}

    \item [Automatiseren infrastructuurbeheer en goed monitoren:] vergroting
    flexibiliteit/reactiviteit door extensieve monitoring belasting,
    automatische opschaling en automatische infrastructuur-configuratie
    door gebruik 'Infrastructure as Code (IaC)'
    \footnote{IaC maakt configuratie-sjablonen zodat inzet nieuwe
    servers wordt geautomatiseerd. \cite{microsoft2023wat}}
    en hulpprogramma's \cite{microsoft2023wat}
    
    \item [Gebruiken cloudproviders:] (automatisch) gebruik flexibiliteit
    cloudinfrastructuur die snel
    servers in verschillende locaties
    kan starten, configureren en afsluiten ('Infrastructure as a Service') 
    \cite{microsoft2023wat, gandhi2018modeldriven, hwang2014scaleout,
    chenhao2017mitigating}

\end{description}

Schadebeperking bij onvermijdbare overbelasting
(bv. selectieve terminatie en admissiecontrole) en maatregelen op
besturingssysteem-/kernelniveau (bv. resource-allocatie)
vallen buiten bestek van deze paper.