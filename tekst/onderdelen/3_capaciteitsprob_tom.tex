% De verschillende secties moeten mooi overgaan in elkaar.
% Hiervoor gebruik je bindteksten. Deze tekstjes zijn typisch de eerste
% paragrafen van de verschillende (sub)secties en leggen uit wat de
% bedoeling van de sectie is, en eventueel hoe dat relateert aan
% de vorige sectie of hoe het past in het grotere geheel.

\section{Capaciteitsproblemen bij overbevraging online diensten}

Bij capaciteitsproblemen zijn de beschikbare 'system resources'
niet in staat om de inkomende 'requests' van gebuikers af te handelen.
Dit kan ertoe leiden dat de systemen traag, slecht (bv. site of applicatie wordt niet correct
weergegeven) of niet (bv. HTTP error codes) functioneren.
Het probleem kan zich constant voordoen, doordat de infrastructuur slecht
functioneert of onvoldoende resources heeft om aan de normale
vraag tegemoet te komen. Anderzijds kan het ook een tijdelijk gegeven zijn,
veroorzaakt door de inadequate flexibiliteit van de systemen om
te reageren op piekbelasting. \newline

Mogelijke oplossing van capaciteitsproblemen zijn de volgende:

\begin{itemize}
    \item Verticaal scalen: upgraden van de hardware van de huidige
    servers (bv. extra geheugen, betere processors, extra bandwijdte)
    \item Horizontaal scalen: voorzien van extra servers, waarover het
    verkeer kan worden verdeeld.
    \item (Betere) load balancing: optimalisatie van de verdeling van het
    verkeer over de verschillende servers (bv. gebruik van betere
    algoritmen)
    \item Hedendaagse DevOps-technieken: DevOps (development operations) kan
    de flexibiliteit vergroten door het automatischeren van deployment/scaling,
    gebruiksmonitoring ene systeemmanagement.
    \item Gebruik van cloud providers: cloud providers beschikking over
    veel infrastructuur voor (automatische) scaling.
    \item Cachen: kan laattijden verminderen.
    \item Geografisch distribueren: vertiaal gescaled servers voldoende
    geografich verspreiden zodat gebrukers over hele wereld sneller
    connectie kunnen maken.
    \item Optialiseren broncode: indien de broncode van de software wordt
    geoptimaliseerd zijn minder 'system resources' nodig voo dezelfde
    taak.
    \item Betere planning: prestatie-evaluaties kunnen helpen om noden in
    te schatten, deze kunn ne gebruik maken van workload simulaties en
    generatoren om goede evaluaties te maken en tijdig aanpassingen door
    te voeren.
    \item Optimalisatie van software-architectuur en -ontwerp:
    bepaalde architecturen en ontwerpen zijn makkelijk te scalen (bv. micro-
    services).
\end{itemize}


TODO: toevoegen referenties (uit notities halen)