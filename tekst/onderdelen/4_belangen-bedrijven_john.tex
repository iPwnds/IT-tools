% De verschillende secties moeten mooi overgaan in elkaar.
% Hiervoor gebruik je bindteksten. Deze tekstjes zijn typisch de eerste
% paragrafen van de verschillende (sub)secties en leggen uit wat de
% bedoeling van de sectie is, en eventueel hoe dat relateert aan
% de vorige sectie of hoe het past in het grotere geheel.

\section{Belangen van ticketsysteembedrijven}



\subsection{Winstmaximalisatie}
Winstmaximalisatie is een cruciaal doel voor bedrijven in de ticketverkoopindustrie, en Ticketmaster is geen uitzondering. Om hun winst te maximaliseren, hanteert Ticketmaster vaak vijf belangrijke factoren bij het bepalen van de prijzen voor tickets. Hieronder worden deze factoren uitgelegd:

\vspace{5 mm}

Face Value Prijs: De face value prijs wordt beschouwd als de prijs die wordt toegepast door zowel de verkoper als de doorverkoper. Dit is een fluctuerende prijs die wordt beïnvloed door zowel vraag als aanbod. Ticketmaster maakt gebruik van dynamische prijsalgoritmen en processen om deze prijs te beïnvloeden op basis van deze factoren. Dit stelt Ticketmaster in staat om de prijs aan te passen aan de veranderende marktomstandigheden.

Service- en Orderverwerkingskosten: Ticketmaster voegt een servicekosten toe op basis van hun overeenkomst/contract met de artiest of de doorverkoper. Dit is een vaste prijs die niet verandert op basis van de nominale waarde van het werkelijke ticket zelf. Ze voegen ook een orderverwerkingskosten toe, die is ontworpen om de kosten van verzending en bescherming te dekken. De verwerkingskosten worden niet in rekening gebracht bij in-person aankopen. Eventuele resterende gelden uit de kosten worden verdeeld tussen de artiest en de verkoper.

Leveringskosten: Ticketmaster past deze prijs aan op basis van de gekozen leveringsmethode. Soms geldt deze vergoeding niet voor bepaalde aankopen. Deze vergoeding wordt extra inkomsten voor Ticketmaster.

Facilitatiekosten: De klant die zijn tickets op Ticketmaster verkoopt, beslist of deze vergoeding wordt toegepast. Deze kosten zijn bedoeld om winst terug te geven aan de daadwerkelijke locatie waar het evenement wordt gehouden. Ticketmaster ontvangt geen winst uit deze vergoeding.

Belastingen: Belastingen vormen over het algemeen geen belangrijke inkomstenbron in vergelijking met het totale inkomstenoverzicht van Ticketmaster. Deze worden bepaald door federale, staats- en lokale wetten en zijn niet onderhandelbaar bij aankoop.

\vspace{5 mm}

Het is duidelijk dat Ticketmaster een uitgebreid prijsmodel hanteert om winstmaximalisatie te bereiken. Door rekening te houden met deze factoren en hun dynamische prijsstrategie kunnen ze de prijzen aanpassen aan veranderende marktomstandigheden, waardoor ze zowel klanten als serviceproviders kunnen bedienen en tegelijkertijd streven naar winstoptimalisatie.

\newpage
\subsection{Klanttevredenheid}

Het e-ticketconcept is aantrekkelijk voor zowel klanten als dienstverleners. Vanuit het perspectief van de klant biedt het de volgende voordelen:
\vspace{5 mm}

A. Voordelen:
\begin{itemize}
    \item Snellere en handigere verificatie van een ticket;
    \item De mogelijkheid om te profiteren van een flexibel tarievenschema, met mogelijke individuele kortingen en speciale aanbiedingen;
    \item Mogelijkheid tot intrekking van verloren tickets en hun vervanging;
    \item Geen noodzaak om contant geld bij zich te hebben, bijvoorbeeld voor een plaatselijke kaartjesautomaat. Dit is vooral handig voor klanten die de vervoersdienst slechts af en toe gebruiken of zich in een andere stad bevinden.
\end{itemize}
\vspace{5 mm}

B. Nadelen:
Ondanks deze voordelen brengt het concept van e-ticketing ook enkele zorgen met zich mee, met name op het gebied van privacy. Deze zorgen omvatten:
\begin{itemize}
    \item Alomtegenwoordige klantidentificatie;
    \item De mogelijkheid van klantprofilering, zoals het creëren van bewegingspatronen;
    \item Potentiële schending van de privacy door verhoogd toezicht, wat vaak wordt aangeduid als "Het Grote Broer"-probleem.
\end{itemize}

\vspace{5 mm}
Kortom, terwijl e-ticketing aanzienlijke voordelen biedt aan klanten, zijn er ook legitieme privacykwesties die moeten worden overwogen en aangepakt in de implementatie van dit systeem. Het handhaven van een balans tussen klanttevredenheid en privacybescherming is van cruciaal belang voor het succes van e-ticketingdiensten.

\subsection{Imago en merkreputatie}

In het streven naar winstmaximalisatie speelt het imago en de merkreputatie een essentiële rol voor bedrijven in de evenementenbranche, zoals Ticketmaster. Een positief imago en een sterke merkreputatie kunnen aanzienlijk bijdragen aan het succes en de winstgevendheid van een ticketbedrijf.

Het imago van Ticketmaster wordt beïnvloed door de manier waarop ze omgaan met de hierboven besproken factoren in hun prijsstrategie. Klanttevredenheid is van cruciaal belang, aangezien tevreden klanten meer geneigd zijn om herhaalaankopen te doen en positieve mond-tot-mondreclame te genereren. Het aanbieden van eerlijke prijzen, duidelijke kostenstructuur en transparante beleidslijnen draagt bij aan een positief imago.

De merkreputatie van Ticketmaster wordt eveneens gevormd door de manier waarop ze omgaan met externe belanghebbenden, zoals artiesten en locaties. Het nakomen van overeenkomsten en het creëren van een win-win-situatie met artiesten en locaties kan leiden tot een sterke merkreputatie en kan de toegang tot exclusieve evenementen vergemakkelijken.

Bovendien moeten bedrijven zoals Ticketmaster rekening houden met de perceptie van het publiek en de media. Negatieve publiciteit met betrekking tot prijzen, transparantie of klantenservice kan schadelijk zijn voor het imago en de merkreputatie van het bedrijf. Het is van cruciaal belang om proactief te reageren op eventuele problemen en klachten om het vertrouwen van klanten en belanghebbenden te behouden.

\vspace{5 mm}
Kortom, in de zoektocht naar winstmaximalisatie in de evenementenbranche is het essentieel om een positief imago en een sterke merkreputatie te cultiveren en te onderhouden. Dit zal niet alleen bijdragen aan de financiële prestaties van het bedrijf, maar ook aan het behoud van klanten en zakelijke partners.

\subsection{Appendix}
Gebruik van genAI om de tekst te schrijven voor de hoofdstukken, waarbij ik een deel van een artikel bij heb geplaats om als bron van informatie te gebruiken. 

nl. https://ieeexplore.ieee.org/abstract/document/7907509, SECTION IV.Discussion.
Ook het bron https://www.kennesaw.edu/coles/centers/markets-economic-opportunity/docs/walton-final.pdf, vanaf pagina 7, de vijf factoren die ticketmaster gebruikt om prijzen te bepalen.
