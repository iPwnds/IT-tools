% De verschillende secties moeten mooi overgaan in elkaar.
% Hiervoor gebruik je bindteksten. Deze tekstjes zijn typisch de eerste
% paragrafen van de verschillende (sub)secties en leggen uit wat de
% bedoeling van de sectie is, en eventueel hoe dat relateert aan
% de vorige sectie of hoe het past in het grotere geheel.


\section{Machtsmisbruik door Grote Spelers}
\label{sec:Hoofdstuk_7}

Ticketsystemen zijn cruciale schakels in de verkoop van toegangskaarten voor evenementen, 
maar ze komen vaak onder vuur te liggen vanwege hun machtspositie en het misbruik daarvan. 
Verschillende aspecten van hun werking onthullen de impact die deze grote spelers hebben op consumenten en de markt in het algemeen.

\subsection{Prijsstelling en Commissies}

Een van de meest controversiële aspecten van ticketsystemen is de prijsstelling en de commissies die ze opleggen. 
Vaak worden tickets verkocht met aanzienlijke extra kosten bovenop de oorspronkelijke prijs, 
waardoor consumenten meer betalen dan ze verwachten. Bovendien innen deze systemen aanzienlijke commissies, 
waardoor zowel artiesten als consumenten zich afvragen of deze kosten gerechtvaardigd zijn.

\subsection{Exclusieve Deals}

Ticketsystemen sluiten vaak exclusieve deals af met zowel locaties als artiesten, 
waardoor ze een monopoliepositie verkrijgen en andere platforms buitensluiten. 
Dit beperkt niet alleen de concurrentie, 
maar belemmert ook de keuze voor consumenten. 
Mensen hebben minder vrijheid om te kiezen waar ze hun tickets willen kopen, 
wat leidt tot een gebrek aan diversiteit in prijzen en diensten.

\subsection{Beperking van Keuze voor Consumenten}

Door exclusieve deals en monopolistische praktijken kunnen consumenten beperkt worden in hun keuzemogelijkheden. 
Ze worden gedwongen om via een specifiek ticketsysteem te kopen, 
zelfs als ze dat niet willen vanwege hogere kosten of slechte ervaringen in het verleden.

\subsection{Data Verzameling en Privacy}

Ticketsystemen verzamelen enorme hoeveelheden gegevens over consumenten, 
vaak zonder transparantie of toestemming. 
Deze gegevens worden gebruikt voor gerichte marketing en soms zelfs doorverkocht aan derden, 
wat ernstige zorgen oproept over privacy en gegevensbeveiliging.

\vspace{10 mm}

Ticketsystemen hebben aanzienlijke macht en hun misbruik daarvan heeft diepgaande gevolgen voor consumenten en de markt. 
Door regulering, 
transparantie en bevordering van concurrentie kan dit misbruik worden aangepakt, 
waardoor de ticketmarkt eerlijker en gunstiger wordt voor zowel artiesten als consumenten.

\cite{Similarweb, Wikimedia, Investopedia}