% De verschillende secties moeten mooi overgaan in elkaar.
% Hiervoor gebruik je bindteksten. Deze tekstjes zijn typisch de eerste
% paragrafen van de verschillende (sub)secties en leggen uit wat de
% bedoeling van de sectie is, en eventueel hoe dat relateert aan
% de vorige sectie of hoe het past in het grotere geheel.

\section{Belangen van klanten/gebruikers}
Ticketsystemen hebben ten opzichte van de gebruiker een aantal voordelen en nadelen. Dit komt ook mede doordat er veel factoren komen kijken bij het maken van ticketsystemen. 
Deze voor- en nadelen kunnen de belangen van een klant toe- en tegenwerken.

\subsection{Voordelen}
De meeste voordelen van zulke ticketsystemen zijn redelijk voor de hand liggend, bv. minder gebruik van papier door deze online tickets.
Maar sommige voordelen maken ticketsystemen nog interesanter voor de gebruiker, zoals de gebruiksvriendelijkheid die deze systemen met zich meebrengen.
Makers van ticketsystemen ontwerpen deze zodanig dat het systeem gemakkelijk en effficiënt te gebruiken is zodat gebruikers zich gemakkelijk kunnen navigeren en tickets kunnen aankopen.
Tickets verschijnen meestal ook direct na aankop in de mailbox van klanten zodat ze zeker zijn van hun aankoop, als dit niet zo verloopt is er nog 
altijd een helpdesk aanwezig bij de meeste ticketsystemen \cite{cm-voordelen}. Gebruikers kunnen meestal 24/7 terecht bij zo een helpdesk, er is hulp ter beschikking als er iets zou foutlopen.
Eens de gebruiker een ticket heeft gekocht kunnen zij ook heel gemakelijk zien wat de status van hun ticket is aan de hand van real-time updates.
Nog een groot voordeel van ticketsystemen is dat de ticketsystemen altijd en overal beschikbaar zijn, online ticketsystemen zijn altijd toegankelijk zolang de gebruiker een verbinding met het internet heeft \cite{Benefitsonline2023}.
Klanten kunnen zelfs nog last minut tickets kopen als deze nog beschikbaar zijn \cite{cm-voordelen}.

\subsection{Nadelen}
Er komen ook nadelen bij ticketsystemen kijken die de belangen van een gebruiker kunnen verhinderen. Ticketsystemen zijn meestal online en het is dus nodig om een stabiele internetverbinding te hebben.
Gebruikers hebben dit misschien niet waardoor er wat fout kan gaan met een verbinding die wegvalt tijdens het aankopen van tickets. Ookal worden ticketsystemen ontworpen voor gemakkelijk gebruik,
hebben gebruikers toch een kleine technische aanleg nodig, online systemen gebruiken is minder voor de hand liggend voor bv. oudere mensen die nog niet bekend zijn met het internet.
Omdat deze ticketsystemen meestal via websites zijn kunnen deze bij een hoog aantal gebruikers crashen, waardoor al de gebruikers het ticketsysteem niet kunnen gebruiken. Dit geldt niet voor alle ticketsystemen, de meeste hebben grote servers die veel bezoekers aankunnen.
Een nadeel dat zich tijdens het betalen van de tickets voordoet, is dat er een fees of commissies kunnen worden opgelegt door de ticketing maatschapij voor sercvice etc., waardoor de prijs kan oplopen, dit wordt besproken in volgend hoofdstuk.
Gebruikers kunnen ook ondervinden dat er soms beperkte betaalmethodes zijn die niet voor de gebruiker werken. Als het betalen dan toch gelukt is, kan het ook zijn dat de gebruiker niet nauwkeurig genoeg is geweest en dat de persoonlijke info op het ticket niet klopt.
Dit kan iedereen wel eens overkomen maar ticketsystemen hebben meestal geen of een slechte refund policy waardoor de klant geld verliest \cite{concert-tickets-online2021}. 

